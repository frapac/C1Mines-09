\documentclass[slidstop,compress,8pt]{beamer}
\usepackage{babel}
\usepackage{marvosym,subfig}
\usepackage{animate}
\usepackage{algorithm2e}
\usepackage{graphics,color,graphics,fancybox,ragged2e,dingbat}
\usepackage{tikz}
\usepackage{float}
\usetikzlibrary{trees}
\useoutertheme[]{default}
\usecolortheme{orchid}%{dolphin} %%colore tema
%\usecolortheme[rgb={.00,.52,.30}]{structure}
%\usecolortheme[rgb={.0,.5,.2}]{structure}
\usefonttheme[sefif,onlymath]{serif} %%font
\usefonttheme[]{structuresmallcapsserif} %%,structurebold
\usetheme{default}


\setbeamercolor*{title}{use=structure,fg=white,bg=structure.fg}
\setbeamertemplate{title page}[default][colsep=-4bp,rounded=true,shadow=true]

\definecolor{bluescuro}{RGB}{10,50,190}
\definecolor{celestino}{RGB}{193,210,243}
\definecolor{mygreen}{rgb}{0.,0.5,0.1}
\setbeamercolor{alerted text}{fg=bluescuro}


%====================================================
\newcommand{\vAS}{[vAckoSag14]}%\cite{vanAckooij_Sagastizabal_2012}\/}
\newcommand{\vAO}{LBM\/}
\newcommand{\SupHyp}{[Prek03]}%{Alg.\cite{Prekopa_2003}\/}
\newcommand{\FeasDir}{[Prek70]}%{Alg.\cite{Prekopa_1970}\/}
\newcommand{\BunCen}{[Kiw08]}%{Alg.\cite{Kiwiel_2008}\/}
\newcommand{\PrimDual}{[LNN95]}%{Alg.\cite{Lemarechal_Nemirovskii_Nesterov_1995}\/}


\def\argmax{\mathop{\rm arg\,max}}
\def\argmin{\mathop{\rm arg\,min}}
\DeclareMathOperator{\dom}{dom}
\DeclareMathOperator{\llev}{lev}
\DeclareMathOperator{\llow}{low}
\DeclareMathOperator{\uup}{up}
\DeclareMathOperator{\lc}{c}
\DeclareMathOperator{\lf}{f}
\DeclareMathOperator{\tar}{tar}
\DeclareMathOperator{\rec}{rec}
%\DeclareMathOperator{\tol}{tol}
\newcommand{\tol}{\delta_{\texttt{Tol}}}
%\newcommand{\eps}{\epsilon}
\newcommand{\Q}{\mathcal{Q}}
\newcommand{\F}{\mathcal{F}}
\newcommand{\E}{\mathbb{E}}
\newcommand{\cvar}{\texttt{AV@R}_\beta}
\newcommand{\inner}[2]{\langle#1,#2\rangle}
\newcommand{\flin}{{\bar{f}}}
\newcommand{\clin}{{\bar{c}}}
\newcommand{\x}{x}
\newcommand{\X}{X}
\newcommand{\Xlev}{\mathbb{X}}
\newcommand{\dsum}{\displaystyle\sum}
\newcommand{\xkk}{{\x^{k+1}}}
\newcommand{\xk}{{\x^{k}}}
\newcommand{\xc}{{\hat{\x}}}
\newcommand{\xr}{\x_{\rec}}
\newcommand{\xrk}{\xr^k}
\newcommand{\xck}{{\xc^k}}
\newcommand{\xckk}{{\xc^{k+1}}}
\newcommand{\fbun}{J_{\lf}}
\newcommand{\cbun}{J_{\lc}}
\newcommand{\fmodel}{\check{f}}
\newcommand{\cmodel}{\check{c}}
%\newcommand{\flow}{f_{\llow}}
%\newcommand{\flev}{f_{\llev}}
\newcommand{\kapl}{\gamma}
\newcommand{\finf}{f_{\min}}
\newcommand{\xinf}{\x_{\min}}
\newcommand{\nn}{\nonumber}
\newcommand{\gf}{g_{\lf}}
\newcommand{\gc}{g_{\lc}}
\newcommand{\agf}{\hat{g}_{\lf}}
\newcommand{\agc}{\hat{g}_{\lc}}
\newcommand{\pf}{\mu_{\lf}}
\newcommand{\pc}{\mu_{\lc}}
\newcommand{\R}{\Re}
\newcommand{\itef}{{a_{\lf}}}
\newcommand{\itec}{{a_{\lc}}}
\newcommand{\ite}{a}
\newcommand{\ferror}{\eta_{\lf}}
\newcommand{\cerror}{\eta_{\lc}}
\newcommand{\ferrorg}{\epsilon_{\lf}}
\newcommand{\cerrorg}{\epsilon_{\lc}}
\newcommand{\error}{\eta}
\newcommand{\ftar}{f_{\tar}}
\newcommand{\ctar}{c_{\tar}}
\newcommand{\fl}{{w}}
\newcommand{\pp}{{\texttt{t}}}
\newcommand{\f}{{\check{f}}}
\newcommand{\nred}{{N_{\texttt{rep}}}}
\newcommand{\Ilp}{{I_{\texttt{rep}}}}
\newcommand{\pula}{\vspace{.1in}}
\newcommand{\azul}[1]{{\color{blue}#1}}
\newcommand{\verde}[1]{{\color{mygreen}#1}}
\newcommand{\verm}[1]{{\color{red}#1}}
\newcommand{\var}{\mathbb{V}\texttt{ar}}
\newcommand{\cov}{\mathbb{C}\texttt{ov}}
\newcommand{\corr}{\mathbb{C}\texttt{orr}}
\newcommand{\prob}{\mathbb{P}}
\newcommand{\flow}{f^{\llow}}
\newcommand{\fup}{f^{\uup}}

\newcommand{\norm}[1]{\ensuremath{\Arrowvert #1 \Arrowvert}}
\definecolor{mygreen}{rgb}{0.,0.5,0.1}
\def\vert#1{{\color{mygreen}#1}}

\newenvironment{retraitsimple}{\begin{list}{--~}{
 \topsep=0.3ex \itemsep=0.3ex \labelsep=0em \parsep=0em
 \listparindent=1em \itemindent=0em
 \settowidth{\labelwidth}{--~} \leftmargin=\labelwidth
}}{\end{list}}

\usebackgroundtemplate{\includegraphics[width=\paperwidth]{../Figs/coverlight}}
\setbeamertemplate{footline}[frame number]



%======================================================================================
\title{Stochastic Programming \newline {\small Stochastic programming with recourse}}

\author{{\bf Fran\c{c}ois Pacaud  and Welington de Oliveira} \\
{\scriptsize \texttt{frapac.github.io}}
}
\institute{\bf Centre Automatique et Systèmes - CAS \\
\'Ecole Nationale Sup\'erieure des Mines de Paris - Mines Paris PSL %\\
}
% CMA - MINES Paris PSL}
%\logo{\includegraphics[width=1.2cm]{../Figs/logo}}
\titlegraphic{\includegraphics[width=5cm]{../Figs/LogoMinesPSL}}



\date{C1MINES-09 PSL week: Optimisation Stochastique, November 2024}

\begin{document}
\frame[noframenumbering,plain]{\titlepage}



\begin{frame}[noframenumbering,plain]{Outline}
\tableofcontents %[pausesections]
\end{frame}

%---------------------------------------------------------------------
\usebackgroundtemplate{\includegraphics[width=\paperwidth]{../Figs/white}}




%
%---------------------------------------------------------------------
\section{SP with recourse: Two-stage Linear Programming - 2SLP}
\usebackgroundtemplate{\includegraphics[width=\paperwidth]{../Figs/coverlight}}
\begin{frame}[noframenumbering,plain]{ }
\begin{block}{\Large Two-stage Linear Programming - 2SLP}
\end{block}
\end{frame}
\usebackgroundtemplate{\includegraphics[width=\paperwidth]{../Figs/white}}


\begin{frame}{Stochastic programming with recourse}
Consider the LP
\[
\left\{\begin{array}{ll}
\displaystyle \min_{(x,y)\geq 0} & c^\top x+
q^\top y\\
 s.t. & Ax =  b \\&Tx+Wy=h
\end{array}\right.
\]
Now suppose that some (or all) the data $q,T,W,h$ depend on some random vector $\omega$:
\[
\verm{q(\omega),\,T(\omega),\,W(\omega),\,h(\omega)}
\]
Decisions are sequential in nature:
\begin{equation*}
  x \leadsto \omega \leadsto y
\end{equation*}
We shall to give a ``meaning" to the random LP:
$q(\omega)^\top y(\omega)$ is now random!
\[
\left\{\begin{array}{ll}
\displaystyle \min_{(x,y(\omega))\geq 0} & c^\top x+
q(\omega)^\top y(\omega)\\
 s.t. & Ax =  b \\&T(\omega) x+W(\omega)y(\omega)=h(\omega)
\end{array}\right.
\]

\pula
\verm{A manner is to minimize the expected cost}
\end{frame}
\begin{frame}{Two-stage stochastic linear programs with recourse - 2SLP}

Minimizing the present cost \verm{\large +} the expected value of the future costs
\pula

\begin{block}{Optimization on average}
\[
\left\{\begin{array}{ll}
\displaystyle \min_{(x,y(\omega))\geq 0} & c^\top x+
\E[q(\omega)^\top y(\omega)]\\
 s.t. & Ax =  b \\&T(\omega) x+W(\omega)y(\omega)=h(\omega) \; a.s.
\end{array}\right.
\]
(a.s. = almost surely)
\end{block}

\begin{itemize}
  \item $x$ represents the \emph{here-and-now} variables \\ (the decisions we have to make in the present)
  \item $y(\omega)$ represents the \emph{wait-and-see} decisions, a.k.a \emph{recourse} \\ (the decisions we have to make in the future, depending on the scenario)
\item $W(\omega)$ is the matrix of recourse, and the matrix of technologies $T(\omega)$ couples the variables $x$ and $y$
\end{itemize}

\end{frame}

\begin{frame}{Finitely many scenarios}
%Consider $N$ scenarios $\xi^i$, $i=1,\ldots, N$, with associated probability $p_i >0$ (e.g. $1/N$)
In two-stage stochastic linear programming problems with finitely many scenarios
\verm{$(q^i,T^i,W^i,h^i)$}, $i=1,\ldots,N$, we wish to solve the high dimensional problem
\begin{block}{Deterministic equivalent}
\[
\left\{
\begin{array}{ll}
\displaystyle \min_{x,y^i} & c^\top x+\sum_{i=1}^N p_i [q^{i\,\top} y^i]\\
\mbox{s.t.} & Ax =  b,\; x\geq 0 \\&T^i x+W^i y^i=h^i,\; y^i\geq 0, \; i=1,\ldots,N\\
\end{array}
\right.
\]

\pula

\pause
\azul{This is a LP with a block-arrowhead structure!}


\[
\left\{
\begin{array}{lllllllllllll}
\min & c^\top x &+ p_1q^{1\,\top} y^1  &+ p_2q^{2\,\top} y^2 &+ \cdots  &+ p_Nq^{N\,\top} y^N \\
\mbox{s.t}& Ax =b\\
&\\
& T^1x           &+ W^1y^1&&&&= h^1 \\
& T^2x           &&+ W^2y^2&&&= h^2 \\
& \vdots         &&&\ddots \\
& T^Nx           &&&&+ W^Ny^N&= h^N \\
&\\
&(x, y)\geq 0
\end{array}
\right.
\]

\end{block}
\end{frame}

\begin{frame}{Some orders of magnitude}
  How fast can we solve a large-scale LP using a state-of-the-art solver?


  \begin{table}
  \begin{tabular}{rrr}
    \hline
    n & m & Solve time (s) \\
    \hline
    85M & 98M & 6,000 \\
    223M & 254M & 66,100 \\
    \hline
  \end{tabular}
  \caption{Time to solve a LP with $n$ variables, $m$ constraints}
  \end{table}

  \vspace{2cm}

  Source:
  \begin{itemize}
    \item Rehfeldt, D., Hobbie, H., Schönheit, D., Koch, T., Möst, D., \& Gleixner, A. (2022).
      \emph{A massively parallel interior-point solver for LPs with generalized arrowhead structure, and applications to energy system models.}
  \end{itemize}


\end{frame}


\begin{frame}{Finitely many scenarios}
In two-stage stochastic linear programming problems with finitely many scenarios
\verm{$(q^i,T^i,W^i,h^i)$}, $i=1,\ldots,N$, we wish to solve the high dimensional problem
\begin{block}{Deterministic equivalent problem}
\[
\left\{
\begin{array}{lllllllllllll}
\min & c^\top x &+ p_1q^{1\,\top} y^1  &+ p_2q^{2\,\top} y^2 &+ \cdots  &+ p_Nq^{N\,\top} y^N \\
\mbox{s.t}& \verm{x \in X, \;y \in Y}  \\
&\\
& T^1x           &+ W^1y^1&&&&= h^1 \\
& T^2x           &&+ W^2y^2&&&= h^2 \\
& \vdots         &&&\ddots \\
& T^Nx           &&&&+ W^Ny^N&= h^N
\end{array}
\right.
\]

\azul{We can have "mixed-integer constraints" in $X$ and $Y$}
\pula

\verm{However, depending on $N$ the deterministic equivalent problem cannot be solved directly...}
\end{block}

\begin{itemize}
\item \# variables: $n_x + N\, n_y$
\item \# constraints: $m_x + N\, n_y$
\end{itemize}


\verm{The deterministic equivalent problem is only useful when $N$ is small enough...}

\end{frame}
%
%\begin{frame}{Computational practice}
%\begin{center}
%\azul{\shadowbox{Power generation planning under uncertainty}}
%\end{center}
%\pula
%Consider the problem available at the link \newline
%{\scriptsize
%\url{https://colab.research.google.com/drive/1mJGxrpmOPnG8cJMlNPxULxj-vUsOg6ql?usp=sharing}
%}
%\pula
%
%\verm{Tasks:}
%\begin{enumerate}
%\item Solve the $N$ scenario subproblems and compute the first-stage solutions $\bar x(\omega^i)$, $i=1,\ldots,N$
%\item Compute the average solution
%\[
%x^{\tt av}= \frac{1}{N}\sum_{i=1}^N \bar x(\omega^i)
%\]
%\item Compute the expected cost of the average solution
%
%\item Compute a solution $x^{\tt SP}$ by solving the equivalent deterministic problem
%
%\item Compare $x^{\tt SP}$, $x^{\tt av}$ and their expected cost
%\end{enumerate}
%
%\end{frame}


\begin{frame}{Two-stage stochastic linear programs with recourse}

\begin{center}
\verm{The deterministic equivalent problem can be too large to be solved directly}
\end{center}

\begin{center}
\shadowbox{We need to decompose the problem}
\end{center}

For convenience of notation, let's
write the 2SLP more compactly as
\[
\left\{\begin{array}{ll}
\displaystyle \min_{(x,y)\geq 0} & c^\top x+
\E[q^\top y]\\
 s.t. & Ax =  b \\&T x+Wy=h \;\;\; a.s.
\end{array}\right.
\]
We define also the random vector
\[
\xi(\omega): = (q(\omega),h(\omega),T(\omega),W(\omega))
\quad \mbox{or simply} \quad
\azul{\xi: = (q,h,T,W)}
\]
and split the problem according to first and second-stage variables

\end{frame}



\begin{frame}{Two-stage decomposition}
  \begin{block}{Value function}
    For each pair \verm{$(x,\xi)$}, define the simple LP
    \[Q(x,\xi):=\left\{ \begin{array}{rll}
        \min & q^\top y &\\
        \mbox{s.t.} & W y   =h-Tx&\\
                    & y  \geq 0  &
    \end{array} \right.\]
  \end{block}
The program 2SLP is equivalent to
\[\left\{ \begin{array}{rll}
\min & c^\top x +\E\left[Q(x,\xi)\right] &\\
\mbox{s.t.} & Ax=b\,, \quad x\geq0\,,&
\end{array} \right.\]

\begin{itemize}
\item For fixed $\tilde\xi$, when is $Q(\cdot,\tilde\xi)$ finite?
\item What does $Q(x,\tilde\xi)=-\infty$ mean?
\only<2>{ \verm{(no solution - depending on $\prob(\tilde \xi)$)}}
\item What does $Q(x,\tilde\xi)=+\infty$ mean?
\only<2>{\verm{(infeasibility!)}}
\end{itemize}



\end{frame}

\begin{frame}{Well-posedness}
  \begin{block}{Fixed recourse}
    The two-stage problem has \emph{fixed} recourse if $W$ does
    not depend on $\omega$.
  \end{block}

  \begin{block}{Complete recourse}
    The two-stage problem has \emph{complete} recourse if the system
    $W y = z$ has a solution for every $z$ (ensure feasibility for all $z := h - Tx$)
  \end{block}

  \begin{block}{Relatively complete recourse}
    The two-stage problem has \emph{relatively complete} recourse
    if for every $x$ in the set $\{ x \; : \; Ax = b \;, \; x \geq 0 \}$
    and for every $\xi \in \Xi$ the feasible set is non-empty:
    \begin{equation*}
     Y(x, \xi) \neq \emptyset
    \end{equation*}
    with
    \begin{equation*}
      Y(x, \xi) = \{ y \; : \; Tx + Wy = h \;, \; y \geq 0 \}
    \end{equation*}
  \end{block}
\end{frame}

\begin{frame}{Properties of the value function}
  The dual of the second-stage problem is
  \begin{equation*}
      \min_{u} \; u^\top (h - Tx) \quad \text{s.t.} \quad W^\top u \leq q
  \end{equation*}

  \begin{block}{Structure}
    \begin{itemize}
      \item The function $Q(\cdot, \xi)$ is convex.
      \item If $\{u \; : \; W^\top u \leq q \}$
        is non-empty and second-stage problem is feasible,\\
        then $Q(\cdot, \xi)$ is \emph{polyhedral}.
    \end{itemize}
  \end{block}

  \begin{block}{Dual reformulation}
    Suppose $Y(x, \xi) \neq \emptyset$. Then
    \begin{equation*}
      Q(x, \xi) = \max_{u} \; u^\top (h - Tx) \quad \text{s.t.} \quad W^\top u \leq q
    \end{equation*}
  \end{block}

  \begin{block}{Subdifferentiability}
    Suppose for $(x, \xi)$ $Q(x, \xi) < +\infty$. Then $Q(\cdot, \xi)$ is subdifferentiable
    at $x$, with
    \begin{equation*}
      \partial Q(x, \xi) = -T^\top \mathcal{D}(x, \xi) \quad
      \text{where} \quad
      \mathcal{D}(x, \xi) := \argmax_{u \in \Lambda(q)} u^\top (h - Tx)
    \end{equation*}
  \end{block}
\end{frame}

\begin{frame}{Two-stage program with finitely many scenarios}
  Suppose we have a finite number of scenarios $\xi^1, \cdots, \xi^N$.
  Let
  \begin{equation*}
    \phi(x) := \mathbb{E}\big[ Q(x, \xi) \big] = \sum_{i=1}^N p_i Q(x, \xi_i)
  \end{equation*}
  The two-stage program is equivalent to
  \[
    \left\{ \begin{array}{rll}
        \min & c^\top x +\sum_{i=1}^N p_i Q(x,\xi_i) &\\
  \mbox{s.t.} & Ax=b\,, \quad x\geq0\,,&
  \end{array} \right.
  \]

  \begin{block}{Proposition}
    Suppose there exists $x_0$ such that $\phi(x_0) < +\infty$.\\
    Then $\phi(\cdot)$ is polyhedral, and for all $x \in \dom(\phi)$,
    \begin{equation*}
      \partial \phi(x) = \sum_{i=1}^N p_k \partial Q(x, \xi_i)
    \end{equation*}
  \end{block}
\end{frame}



%\begin{frame}{Two-stage decomposition}
%
%\begin{itemize}
%\item In practice, we need to generate a sample of scenarios $\{\xi^1, \ldots, \xi^N\}$, with scenario probability $p_i$,  from the distribution of $\xi$ and approximate
%\[
%\E[Q(x,\xi)] \quad \mbox{ by }\quad \sum_{i=1}^N p_i Q(x,\xi^i)
%\]
%\pula
%
%\item Notice that $Q(x,\xi)$ (for a fixed $x \in {\tt dom} \,Q$) is a random variable as well
%\pula
%
%\item  \azul{The strong law of large numbers} ensures that
%
%\[
%\lim_{N \to \infty} \frac{1}{N}\sum_{i=1}^N Q(x,\xi^i) = \E[Q(x,\xi)]
%\]
%\verde{with probability 1 - w.p.1}
%
%\pula
%
%\item This result justifies why we can replace $\E[Q(x,\xi)]$ with $\sum_{i=1}^N p_i Q(x,\xi^i)$: if your random scenario generator is good enough, just pick up a ``large enough" $N$ and set $p_i=1/N$
%
%\end{itemize}
%\end{frame}

%

\section{SP with recourse: The L-Shaped Method}
%---------------------------------------------------------------------
\usebackgroundtemplate{\includegraphics[width=\paperwidth]{../Figs/coverlight}}
\begin{frame}[noframenumbering,plain]{ }
\begin{center}
\shadowbox{The L-Shaped Method}
\end{center}
\end{frame}
\usebackgroundtemplate{\includegraphics[width=\paperwidth]{../Figs/white}}
%\input{Benders}

%----------------

\begin{frame}{Decomposition}
  We aim at solving the two-stage program:
  \[
    \left\{ \begin{array}{rll}
        \min & c^\top x +\sum_{i=1}^N p_i \verm{\only<1-2>{Q(x,\xi_i)}\only<3->{r_i}} &\\
  \mbox{s.t.} & Ax=b\,, \quad x\geq0\,,&
\only<3->{\\ & \verm{r_i \geq \alpha_k^i + \beta_k^i x} \quad \forall k=1, \cdots, V \quad \forall i=1,\cdots,k }
  \end{array} \right.
  \]
  Reformulation:
  \begin{itemize}
    \item<2-> $\verm{Q(\cdot, \xi)}$ is polyhedral
      \begin{equation*}
        Q(x, \xi) = \max_{k=1,\cdots,V} \; \{ \beta_k x + \alpha_k \}
      \end{equation*}
    \item<3-> $\verm{Q(\cdot, \xi)}$ is convex
      \begin{equation*}
        \min_{x}\; Q(x, \xi) \quad \equiv\quad  \min_{x, r}\;  r \quad \text{s.t.} \quad r \geq \beta_k x + \alpha_k \quad \forall k=1, \cdots, V
      \end{equation*}
  \end{itemize}
\end{frame}


\begin{frame}{Decomposition: multi-cut version}
Given $x^k$, The Benders' decomposition computes a vertex by solving, for $i=1, \cdots, N$,
\[
  \text{\azul{(Lower problem)}}\quad  u_i^k\in\argmax_{u} \; (h_i- T_i x^k)^\top u  \quad \mbox{s.t.} \quad
W_i^\top u \leq q_i
\]
and updates $(\xkk, r^{k+1})$ by solving the LP
\[
\text{\azul{(Upper problem)}}\quad
\left\{
\begin{array}{llll}
  \displaystyle \min_{x,r_1, \cdots, r_N} &c^\top x + \sum_{i=1}^N p_i r_i \\
\mbox{s.t.} & Ax = b \;, \; x\geq 0\\
            & (h_i- T_i x)^\top u_i^\ell \leq r_i & i=1,\ldots,N \quad \forall \ell=1,\cdots, \verm{k}
\end{array}
\right.
\]
\end{frame}


\begin{frame}{Decomposition: single-cut version}

The Upper Problem
\[
\text{\azul{(Upper problem)}}\quad
\left\{
\begin{array}{llll}
  \displaystyle \min_{x,r_1, \cdots, r_N} &c^\top x + \sum_{i=1}^N p_i r_i \\
\mbox{s.t.} & Ax = b \;, \; x\geq 0\\
            & (h_i- T_i x)^\top u_i^\ell \leq r_i & i=1,\ldots,N \quad \forall \ell=1,\cdots, \verm{k}
\end{array}
\right.
\]
is equivalent to
\[
 \left\{
\begin{array}{llll}
\displaystyle \min_{x,r} &c^\top x + r\\
\mbox{s.t.} & Ax = b \;,\; x\geq 0\\
             & \sum_{i=1}^N p_i({h_i}- {T_i}x)^\top u_i^\ell \leq r & \forall \ell=1,\ldots,\verm{k}
\end{array}
\right.
\]
with
\[
u_i^\ell \in\argmax_u \;({h_i}- {T_i}x^\ell)^\top {u}  \quad \mbox{s.t.} \quad
{W_i}^\top u \leq q_i
\]
\end{frame}

\begin{frame}{Decomposition}

\verm{The Benders' decomposition applied to the LP}
\[
\left\{
\begin{array}{lllllllllllll}
\min & c^\top x &+ p_1q^{1\,\top} y^1  &+ p_2q^{2\,\top} y^2 &+ \cdots  &+ p_Nq^{N\,\top} y^N \\
\mbox{s.t}& Ax =b\\
&\\
& T^1x           &+ W^1y^1&&&&= h^1 \\
& T^2x           &&+ W^2y^2&&&= h^2 \\
& \vdots         &&&\ddots \\
& T^Nx           &&&&+ W^Ny^N&= h^N \\
&\\
&(x, y)\geq 0
\end{array}
\right.
\]
\verm{is known as the {\bf L-Shaped method}}
\end{frame}

\begin{frame}{Benders' decomposition for 2SLP}

\begin{block}{The L-Shaped Method}
Given $x^1$ feasible, set $k=1$ and  $UB^0=+\infty$
\begin{enumerate}
\item Send $\xk$ to the Lower Problems: \verm{for $i=1,\ldots,N$, compute a new vertex $u_i^k$ by solving}
\[
\text{\azul{(Lower Problem)}}\quad
Q(\verm{\xk}, \xi_i)=\left\{
\begin{array}{llll}
\displaystyle \max_{u} &(h_i- T_i\verm{\xk})^\top u\\
\mbox{s.t.} & W_i^\top u \leq q_i
\end{array}
\right.
\]
Set $\mathbf{Q}(\xk) = \sum_{i=1}^N p_s Q(\xk, \xi_i)$ and  $UB^k=\min\{UB^{k-1},c^\top \xk +\mathbf{Q}(\xk)\}$
\only<2>{
   and
\[\mbox{\verm{$\beta^{k} = -\sum_{i=1}^N p_i[ T_i^\top u_i^k ]$ and $\alpha^{k}= \sum_{i=1}^N p_i[h_i^\top u_i^k ]$}}
\]
}

\item Find $\verm{(\xkk, r^{k+1})}$ by solving the LP
\[
\text{\azul{(Upper problem)}}\quad
\left\{
\begin{array}{llll}
\displaystyle \min_{x,r} &c^\top x + r\\
\mbox{s.t.} & Ax = b \; , \; x \geq 0 \\
\only<1>{& \sum_{s=1}^Np_s({h^s}- {T^s}x)^\top {u^{i,s}} \leq r& i=1,\ldots,\verm{k}}
\only<2>{&\verm{ \beta^{i\, \top}}x + \verm{\alpha^i} \leq r& i=1,\ldots,k}
\end{array}
\right.
\]

\item If $UB^k - [c^\top x^{k+1}+r^{k+1}]\leq \tol$, stop

\item Set \azul{$k=k+1$} and go back Step 1
\end{enumerate}
\end{block}

\verde{The algorithm stops after finitely many steps (even if $\tol=0$) at the solution without enumerating all the vertices}

\end{frame}



\section{SP with recourse: The Progressive-Hedging Method}
%---------------------------------------------------------------------
\usebackgroundtemplate{\includegraphics[width=\paperwidth]{../Figs/coverlight}}
\begin{frame}[noframenumbering,plain]{ }
\begin{center}
  \shadowbox{Progressive Hedging}
\end{center}
\end{frame}
\usebackgroundtemplate{\includegraphics[width=\paperwidth]{../Figs/white}}
%\input{Benders}

\begin{frame}{Dual Decomposition}
  We remind the two-stage program:
  \[
    \left\{ \begin{array}{rll}
        \min_x & c^\top x +\sum_{i=1}^N p_i Q(x,\xi_i) &\\
  \mbox{s.t.} & Ax=b\,, \quad x\geq0\,,&
  \end{array} \right.
  \]
  Reformulation:
  \begin{equation*}
    \left\{ \begin{array}{rll}
        \min_x & \sum_{i=1}^N p_i \Big(c^\top x + Q(x,\xi_i)\Big) &\\
  \mbox{s.t.} & Ax=b\,, \quad x\geq0\,,&
  \end{array} \right.
  \end{equation*}
  or equivalently, using a splitted formulation:
  \begin{equation*}
    \left\{ \begin{array}{rll} \min_{\{x_i\}_i} & \sum_{i=1}^N p_i \Big(c^\top x_i + Q(x_i,\xi_i)\Big) &\\
  \mbox{s.t.} & Ax_i=b\,, \quad x_i\geq0\,, \\
              & x_i = \verm{\only<1>{x_j}\only<2>{\sum_{j=1}^N p_j x_j}} \quad \forall i=1, \cdots, N
  \end{array} \right.
  \end{equation*}
\end{frame}

\begin{frame}{Dualization of the coupling constraints}
  By dualizing each coupling constraints with a multiplier $\verm{\lambda_i}$:
  \begin{equation*}
    \left\{ \begin{array}{rll} \min_{\{x_i\}_i} \max_{\verm{\lambda}} & \sum_{i=1}^N p_i \Big(c^\top x_i +
        Q(x_i,\xi_i)  + \verm{\lambda_i} \big(x_i - \sum_{j=1}^N p_j x_j \big)\Big) \\
  \mbox{s.t.} & Ax_i=b\,, \quad x_i\geq0\,,\quad \forall i =1,\cdots,N
  \end{array} \right.
  \end{equation*}
  Note that:
  \begin{equation*}
    \begin{aligned}
    \sum_{i=1}^N p_i \verm{\lambda_i} \big(x_i - \sum_{j=1}^N p_j x_j \big)
    &= \sum_{i=1}^N p_i \verm{\lambda_i} x_i - \sum_{i=1}^N \sum_{j=1}^N p_i p_j \verm{\lambda_i} x_j \\
    &= \sum_{i=1}^N p_i \verm{\lambda_i} x_i - \sum_{j=1}^N \sum_{i=1}^N \big( p_i \verm{\lambda_i} \big) p_j  x_j \\
    &= \sum_{i=1}^N p_i \verm{\lambda_i} x_i - \sum_{j=1}^N \mathbb{E}(\verm{\lambda}) p_j  x_j \\
    \end{aligned}
  \end{equation*}
  The problem is equivalent to
  \begin{equation*}
    \left\{ \begin{array}{rll} \min_{\{x_i\}_i} \max_{\verm{\lambda}} & \sum_{i=1}^N p_i \Big(c^\top x_i +
          Q(x_i,\xi_i)  + \big(\verm{\lambda_i} - \verm{\mathbb{E}(\lambda)}\big) x_i \Big) \\
  \mbox{s.t.} & Ax_i=b\,, \quad x_i\geq0\,,\quad \forall i =1,\cdots,N
  \end{array} \right.
  \end{equation*}

\end{frame}

\begin{frame}{Dual problem}
  The dual problem reads
  \begin{equation*}
    \left\{ \begin{array}{rll} \max_{\verm{\lambda}}\min_{\{x_i\}_i}  & \sum_{i=1}^N p_i \Big(c^\top x_i +
          Q(x_i,\xi_i)  + \big(\verm{\lambda_i} - \verm{\mathbb{E}(\lambda)}\big) x_i \Big) \\
  \mbox{s.t.} & Ax_i=b\,, \quad x_i\geq0\,,\quad \forall i =1,\cdots,N
  \end{array} \right.
  \end{equation*}
  For $\verm{\lambda_i}$ given, the inner problem decomposes in $N$ deterministic problems
  \begin{equation*}
    \begin{aligned}
      \min_{x_i} \; & c^\top x_i + Q(x_i, \xi_i) + \big(\verm{\lambda_i} - \verm{\mathbb{E}(\lambda)}\big)x_i \\
      \mbox{s.t.}~ & A x_i = b\, , \quad x_i \geq 0
    \end{aligned}
  \end{equation*}

  \begin{block}{Price of information}
    Any multiplier $\verm{\lambda}$ satisfying the KKT conditions of the
    two-stage problem satisfies
    \begin{equation*}
      \verm{\mathbb{E}(\lambda) = 0}
    \end{equation*}
  \end{block}
  Subproblem is equivalent to
  \begin{equation*}
    \begin{aligned}
      \min_{x_i} \; & c^\top x_i + Q(x_i, \xi_i) + \verm{\lambda_i}x_i \\
      \mbox{s.t.}~ & A x_i = b\, , \quad x_i \geq 0
    \end{aligned}
  \end{equation*}

\end{frame}

\begin{frame}{Dual decomposition algorithm}
Set an initial multiplier $\lambda^0$ such that $\mathbb{E}(\lambda^0) = 0$
  \begin{enumerate}
    \item Solve for each scenario
      \begin{equation*}
        \begin{aligned}
          \min_{x_i} \; & c^\top x_i + Q(x_i, \xi_i) + \verm{\lambda_i^k}x_i \\
          \mbox{s.t.}~ & A x_i = b\, , \quad x_i \geq 0
        \end{aligned}
      \end{equation*}
    \item Update the first-stage variable
      \begin{equation*}
        \overline{x}^{k+1} = \sum_{i=1}^N p_i x_i^{k+1}
      \end{equation*}
    \item Update the price of information as
      \begin{equation*}
        \lambda^{k+1}_i = \lambda^k_i + \rho (x_i^{k+1} - \overline{x}^{k+1})
      \end{equation*}
  \end{enumerate}


\end{frame}

\begin{frame}{Progressive Hedging algorithm}
Set an initial multiplier $\lambda^0$ such that $\mathbb{E}(\lambda^0) = 0$
  \begin{enumerate}
    \item Solve for each scenario
      \begin{equation*}
        \begin{aligned}
          \min_{x_i} \; & c^\top x_i + Q(x_i, \xi_i) + \verm{\lambda_i^k}x_i + \verm{\rho \|x_i - \overline{x}^k \|^2 }\\
          \mbox{s.t.}~ & A x_i = b\, , \quad x_i \geq 0
        \end{aligned}
      \end{equation*}
    \item Update the first-stage variable
      \begin{equation*}
        \overline{x}^{k+1} = \sum_{i=1}^N p_i x_i^{k+1}
      \end{equation*}
    \item Update the price of information as
      \begin{equation*}
        \lambda^{k+1}_i = \lambda^k_i + \rho (x_i^{k+1} - \overline{x}^{k+1})
      \end{equation*}
  \end{enumerate}

  \begin{block}{Convergence}
    Assume that for all $i=1, \cdots, N$, there exists $x_i$ such that $c^\top x_i + Q(x_i, \xi_i) < +\infty$
    with $Ax_i = b$, $x_i \geq 0$. \\
    Then the progressive hedging algorithm converges toward an optimal primal solution
    and the price of information converges toward an optimal price of information
  \end{block}

\end{frame}



%
%%---------------------------------------------------------------------
\section{SP with recourse: Multistage Linear Programming - MSLP}
\usebackgroundtemplate{\includegraphics[width=\paperwidth]{../Figs/coverlight}}
\begin{frame}[noframenumbering,plain]{ }
\begin{block}{\Large Multistage Stochastic Linear Programming - MSLP}
\end{block}
\end{frame}
\usebackgroundtemplate{\includegraphics[width=\paperwidth]{../Figs/white}}
\newcommand{\D}{\mathcal{D}}
%%%%
%\begin{frame}{Nested formulation}
%
%{\scriptsize
%\[
%\min_{x_1\in \X_1} f_1(x_1)+ \E\left[ \min_{x_2 \in \X_2(x_1,\xi_2)}f_2(x_2,\xi_2)+ \E \left[\cdots+\E  [ \min_{x_T\in \X(x_{T-1},\xi_T)}f_T(x_T,\xi_T) ]\right]\right]
%\]
%}
%
%\begin{itemize}
%\item $\xi=(\xi_1,\ldots,\xi_T)$ is the stochastic process
%\pula
%\item \verm{$f_t: \Re^{n_t}\times \Re^{d_t}\to \bar \Re$, $t=1,\ldots,T$, are continuous functions}
%\pula
%\item $x_t \in \Re^{n_t}$, $t=1,\ldots,T$, are the decision variables
%\pula
%\item \verm{$\X_t : \Re^{n_{t-1}}\times \Re^{d_t} \rightrightarrows \Re^{n_t}$,$t=1,\ldots,T$, are measurable, closed valued multifunctions}
%
%\end{itemize}
%
%\end{frame}
%
%%%%%
%\begin{frame}{Nested formulation}
%
%{\scriptsize
%\[
%\min_{x_1\in \X_1} f_1(x_1)+ \E\left[ \min_{x_2 \in \X_2(x_1,\xi_2)}f_2(x_2,\xi_2)+ \E \left[\cdots+\E  [ \min_{x_T\in \X(x_{T-1},\xi_T)}f_T(x_T,\xi_T) ]\right]\right]
%\]
%}
%
%\begin{itemize}
%\item $\xi=(\xi_1,\ldots,\xi_T)$ is the stochastic process
%\pula
%\item \azul{$f_t(x_t,\xi_t):= c_t^\top x_t + i_{\Re^n_+}(x_t)$, $t=1,\ldots,T$}
%\pula
%\item $x_t \in \Re^{n_t}$, $t=1,\ldots,T$, are the decision variables
%\pula
%\item \azul{$\X_t(x_{t-1},\xi_t) :=\{x_t \in \Re^{n_{t-1}}:B_tx_{t-1} + A_tx_t = b_t\}$,} \\\azul{$t=2,\ldots,T$}
%\end{itemize}
%
%
%\end{frame}
%
\begin{frame}{Multistage stochastic linear programs - T-SLP}

\begin{block}{Nested formulation}
{\scriptsize
\[
\min_{\underset{x_1\ge 0}{A_1x_1=b_1}} c_1^\top  x_1+ \E\left[ \min_{\underset{x_2\ge 0}{B_2x_1 + A_2x_2=b_2}}c_2^\top x_2 + \E \left[\cdots+\E   [ \min_{\underset{x_T\ge 0}{B_Tx_{T-1} + A_Tx_T=b_T}}c_T^\top  x_T ]\right]\right]
\]
}
\end{block}
\pula
\begin{itemize}
\item Some elements of the data $\xi=(c_t,B_t,A_t,b_t)$ depend on uncertainties
\end{itemize}

\end{frame}


\begin{frame}{Multistage stochastic linear programs - T-SLP}
\begin{block}{Nested formulation}
{\scriptsize
\[
\min_{\underset{x_1\ge 0}{A_1x_1=b_1}} c_1^\top  x_1+ \E_{\azul{|\xi_{1}}}\left[ \min_{\underset{x_2\ge 0}{B_2x_1 + A_2x_2=b_2}}c_2^\top x_2 + \E_{\azul{|\xi_{[2]}}} \left[\cdots+\E_{\azul{|\xi_{[T-1]}}}   [ \min_{\underset{x_T\ge 0}{B_Tx_{T-1} + A_Tx_T=b_T}}c_T^\top  x_T ]\right]\right] .
\]
}
\end{block}
\pula
\begin{itemize}
\item Some elements of the data $\xi=(c_t,B_t,A_t,b_t)$ depend on uncertainties
\end{itemize}
\end{frame}

\begin{frame}{Scenario tree}
  \tikzstyle{level 1}=[level distance=3.5cm, sibling distance=3.5cm]
  \tikzstyle{level 2}=[level distance=3.5cm, sibling distance=2cm]
  % Define styles for bags and leafs
  \tikzstyle{n1} = [circle, minimum width=10pt,fill, inner sep=0pt]

  % The sloped option gives rotated edge labels. Personally
  % I find sloped labels a bit difficult to read. Remove the sloped options
  % to get horizontal labels.
  \begin{tikzpicture}[grow=right, sloped]
    \node[n1] {}
    child {
      node[n1] {}
      child {
        node[n1, label=right:
        {$x_{2,4}$}] {}
        edge from parent
        node[above] {$p_{24}$}
      }
      child {
        node[n1, label=right:
        {$x_{2,3}$ }] {}
        edge from parent
        node[above] {$p_{23}$}
      }
      edge from parent
      node[above] {$p_{12}$}
    }
    child {
      node[n1] {}
      child {
        node[n1, label=right:
        {$x_{2,2}$ }] {}
        edge from parent
        node[above] {$p_{22}$}
      }
      child {
        node[n1, label=right:
        {$x_{2,1}$ }] {}
        edge from parent
        node[above]  {$p_{21}$}
      }
      edge from parent
      node[above] {$p_{11}$}
    };
  \end{tikzpicture}


\end{frame}


%%%%%%%%%%%%%
%######################################
% Inicio comentario

%%%

\begin{frame}{Scenario trees}

\begin{itemize}
\item Assume that the stochastic process $\xi=(\xi_1,\ldots,\xi_T)$ has a finite number $K$ of realizations

\item Each realization (sequence) is called a scenario $\xi^i=(\xi_1^i,\ldots,\xi_T^i)$

\item Each scenario $\xi^i=(\xi_1^i,\ldots,\xi_T^i)$ has a probability $p_i>0$ associated

\item The value of a given scenario $\xi^i$ at stage $t$ is
denoted a node of the tree
\item The set of all nodes at stage $t$ is denoted by $\Omega_t$
%\item $\Omega_1$ contains only the \emph{root node}
\item The total number of scenarios is $K= |\Omega_T|$
\item We sometimes use the short hand $\iota$ to denote a node: $\iota \in
\Omega_t$
%\item Each scenario $\xi^i$ has a unique node at stage $t=1,2,\ldots,T$

\item The ancestor of a node $\iota \in \Omega_t$ is $a(\iota) \in\Omega_{t-1}$
%(the root node does have a ancestor)

\item The set of descendants (children) of a node $\iota \in \Omega_t$ is denoted by $C_\iota$
\item $\Omega_{t+1} = \cup_{\iota \in \Omega_t} C_\iota$, and $C_\iota \cap C_{\iota '} =\emptyset$ if $\iota \neq\iota '$
%\item Nodes in $\Omega_T$ does have children. Such nodes are called leave nodes.
\item $\mathcal{S}^{(\iota)}$ is the set of all scenarios passing through node $\iota$

\item The probability of node $\iota$ is $p^{(\iota)}:=\prob[\mathcal{S}^{(\iota)}]$

\item Conditional probability $\rho_{a\iota} = \frac{p^{(\iota)}}{p^{(a)}}$ if $a = a(\iota)$
\item  The probability of reaching a node $\iota\in \Omega_t$ is $p^{(\iota)}=\rho_{\iota_1\iota_2}\rho_{\iota_2\iota_1}\ldots \rho_{\iota_{t-1}\iota_t} = \prob[\mathcal{S}^{(\iota)}]$
\end{itemize}
\end{frame}
%#######################################################
%%%

\begin{frame}{Nested formulation}
\begin{block}{Linear case}
{\scriptsize
\[
\min_{\underset{x_1\ge 0}{A_1x_1=b_1}} c_1^\top  x_1+ \E\left[ \min_{\underset{x_2\ge 0}{B_2x_1 + A_2x_2=b_2}}c_2^\top x_2 + \E \left[\cdots+\E   [ \min_{\underset{x_T\ge 0}{B_Tx_{T-1} + A_Tx_T=b_T}}c_T^\top  x_T ]\right]\right]
\]
}
\end{block}
\begin{block}{Linear case + scenario tree}
{\tiny
\if{
\[
\!\!\!\!\!\!\!\!\!\!\!\!\!\!\!\!\!\!\!\!\!\!
%\min_{\underset{x_1\ge 0}{A_1x_1=b_1}} c_1^\top  x_1+ \sum_{\iota_2 \in \Omega_2}p^{\iota_2}\left[ \min_{\underset{x_2\ge 0}{B_2x_1 + A_2x_2=b_2}}c_2^\top x_2 + \sum_{\iota_3 \in \Omega_3}p^{(\iota_3)} \left[\cdots+\sum_{\iota_T \in \Omega_T} p^{(\iota_T)}  [ \min_{\underset{x_T\ge 0}{B_Tx_{T-1} + A_Tx_T=b_T}}c_T^\top  x_T ]\right]\right]
\min_{\underset{x_1\ge 0}{A_1x_1=b_1}} c_1^\top  x_1+ \sum_{\iota_2 \in \Omega_2}\rho_{1\iota_2}\left[ \min_{\underset{x_2\ge 0}{B_2x_1 + A_2x_2=b_2}}c_2^\top x_2 + \sum_{\iota_3 \in \Omega_3}\rho_{\iota_2\iota_3} \left[\cdots+\sum_{\iota_T \in \Omega} \rho_{\iota_{T-1}\iota_T}  [ \min_{\underset{x_T\ge 0}{B_Tx_{T-1} + A_Tx_T=b_T}}c_T^\top  x_T ]\right]\right]
\]
%
%\pause
%\azul{Ops! I forgot to name the nodes...}
}\fi
\[
\begin{array}{l}
\displaystyle\min_{\underset{x_1\ge 0}{A_1x_1=b_1}} c_1^\top  x_1+ \displaystyle \sum_{\iota_2 \in \Omega_2}\rho_{1\iota_2}\left[ \displaystyle \min_{\underset{x_2\ge 0}{B_2^{\iota_2}x_1 + A_2^{\iota_2}x_2=b_2^{\iota_2}}}{c_2^{\iota_2}}^\top x_2 + \displaystyle\sum_{\iota_3 \in \Omega_3}\rho_{\iota_2\iota_3} \Biggl[\cdots+\right.  \\
%
\quad \quad \quad \ldots +\left. \left.\displaystyle\sum_{\iota_T \in \Omega} \rho_{\iota_{T-1}\iota_T}  \left[ \displaystyle\min_{\underset{x_T\ge 0}{B_T^{\iota_T}x_{T-1} + A_T^{\iota_T}x_T=b_T^{\iota_T}}}{c_T^{\iota_T}}^\top  x_T \right]\right]\right]
\end{array}
\]
}
\end{block}
\end{frame}

%
\begin{frame}{Equivalent deterministic}
Denoting $\xi_t^{\iota_i} = (c_t^{\iota_i},B_t^{\iota_i},A_t^{\iota_i},b_t^{\iota_i})$ we can rewrite the above problem as

{\tiny
\[
\!\!\!\!\!\!\!\!\!\!\!\!\!\!\!\!\!\!\!\!\!\!
\left\{
\begin{array}{llllllll}
 \min & \displaystyle c_1^\top  x_1 &+& \sum_{\iota_2 \in \Omega_2}p^{(\iota_2)} {c_2^{\iota_2}}^\top x_2^{\iota_2} &+& \sum_{\iota_3 \in \Omega_3}p^{(\iota_3)} {c_3^{\iota_3}}^\top x_3^{\iota_3} &+ \cdots+&\sum_{\iota_T \in \Omega} p^{(\iota_T)}  {c_T^{\iota_T}}^\top  x_T^{\iota_T} \\
\mbox{s.t.} & A_1x_1 &&&&&&=b_1\\
&B_2^{\iota_2} x_1 &+& A_2^{\iota_2}x_2^{\iota_2}  &&&&= b_2^{\iota_2}  \; \forall \iota_2 \in \Omega_2\\
%
&&&B_3^{\iota_3} x_2^{a(\iota_3)} &+& A_3^{\iota_3}x_3^{\iota_3}  &&= b_3^{\iota_3}  \; \forall \iota_3 \in \Omega_3\\
&&& \ddots && \ddots\\
&&&B_T^{\iota_T} x_{T-1}^{a(\iota_T)} &+& A_T^{\iota_T}x_T^{\iota_T}  &&= b_T^{\iota_T}  \; \forall \iota_T \in \Omega_T
\end{array}
\right.
\]
}
\azul{This is a LP!}
\pause

Example:
\begin{itemize}
\item Suppose $T=12$, each node $\iota_t \in \Omega_t$ has 4 children
\end{itemize}
This gives $4^{11} = 4,194,304$ scenarios.
\begin{itemize}
\item Suppose each $x_t \in \Re^{100}$, $t=1,\ldots,12$
\end{itemize}
\pula
\verm{The number of variables of the above problem is approximately $4.59\times 10^8$!}
\end{frame}

\begin{frame}{Dynamic programming formulation}
\begin{itemize}
\item Stage $t=T$
\[
Q_T(x_{T-1},\xi_{[T]}^\iota):= \min_{\underset{x_T\ge 0}{B_T^\iota x_{T-1}^{a(\iota)} + A_T^\iota x_T=b_T^\iota}}{c_T^\iota}^\top  x_T
\]
\pula
\pause
\item At stages $t=2,\ldots,T-1$
\[
Q_t(x_{t-1},\xi_{[t]}^\iota):= \min_{\underset{x_t\ge 0}{B_t^\iota x_{t-1}^{a(\iota)} + A_t^\iota x_t=b_t^\iota }} {c_t^\iota}^\top  x_t +
\sum_{j \in C_\iota} p^{(j)}\left[Q_{t+1}(x_t,\xi_{[t+1]}^j)\right]
\]
\pula
\pause
\item Stage $t=1$
\verm{
\[
\min_{\underset{x_1\geq 0}{A_1x_1=b_1}} c_1^\top x_1 + \sum_{\iota \in C_1}p^{(\iota)}\left[Q_2(x_1,\xi_2^\iota)\right]
\]
}
\end{itemize}
\end{frame}


\begin{frame}{Dynamic programming formulation}
\begin{itemize}
\item Stage $t=T$
\[
Q_T(x_{T-1},\xi_{[T]}^\iota):= \min_{\underset{x_T\ge 0}{B_T^\iota x_{T-1}^{a(\iota)} + A_T^\iota x_T=b_T^\iota}}{c_T^\iota}^\top  x_T
\]
\pula

\item At stages $t=2,\ldots,T-1$
\[
\underline{Q_t}(x_{t-1},\xi_{[t]}^\iota):= \min_{\underset{x_t\ge 0}{B_t^\iota x_{t-1}^{a(\iota)} + A_t^\iota x_t=b_t^\iota }} {c_t^\iota}^\top  x_t +
\sum_{j \in C_\iota} p^{(j)}\left[\underline{ Q_{t+1}}(x_t,\xi_{[t+1]}^j)\right]
\]
\pula
\item Stage $t=1$
\verm{
\[
\min_{\underset{x_1\geq 0}{A_1x_1=b_1}} c_1^\top x_1 + \sum_{\iota \in C_1}p^{(\iota)}\left[\underline{ Q_2}(x_1,\xi_2^\iota)\right]
\]
}
\end{itemize}
\end{frame}


\begin{frame}{Dynamic programming formulation}
\begin{itemize}
\item Stage $t=T$
\[
Q_T(x_{T-1},\xi_{[T]}^\iota):= \min_{\underset{x_T\ge 0}{B_T^\iota x_{T-1}^{a(\iota)} + A_T^\iota x_T=b_T^\iota}}{c_T^\iota}^\top  x_T
\]
\pula

\item At stages $t=2,\ldots,T-1$
\[
\underline{Q_t}(x_{t-1},\xi_{[t]}^\iota):= \min_{\underset{x_t\ge 0}{B_t^\iota x_{t-1}^{a(\iota)} + A_t^\iota x_t=b_t^\iota }} {c_t^\iota}^\top  x_t +\check{\Q}_{t+1}(x_t,\xi_{[t]}^{\iota})
\]
%
\[
\check{\Q}_{t+1}(x_{t},\xi_{[t]}^\iota):= \sum_{j \in C_{\iota}}p^{(j)}\left[\underline{Q_{t+1}}(x_{t},\xi_{[t+1]}^j)\right]
\]
\pula
\item Stage $t=1$
\verm{
\[
\min_{\underset{x_1\geq 0}{A_1x_1=b_1}} c_1^\top x_1 + \check{\Q}_{2}(x_2,\xi_{[1]})
\]
}
\end{itemize}
\azul{Cutting-plane approximation}
\end{frame}
%%%%%%%%%%%%%%%%
\begin{frame}{Assumptions}
\begin{itemize}
\item The set of nodes $\Omega_t$ has finitely many elements
\pula
\item the problem has relatively complete recourse

\end{itemize}
\pula
\azul{The last hypotheses is made only for sake of simplicity!}
\end{frame}
%%%%
\begin{frame}{Cutting-plane approximation}
\begin{itemize}
\item Stage $t=T$
\[
Q_T(x_{T-1}^k,\xi_{[T]}):= \min_{\underset{x_T\ge 0}{B_T x_{T-1}^{k} + A_T x_T=b_T}}{c_T}^\top  x_T
\]
\pula

\item At stages $t=2,\ldots,T-1$
\[
\azul{\underline{Q_t}(x_{t-1}^k,\xi_{[t]})}:=
\left\{ \begin{array}{llll}
\displaystyle\min_{x_t\geq 0,r_{t+1}} & {c_t}^\top  x_t +r_{t+1} \\
\mbox{s.t.} & B_t x_{t-1}^{k} + A_t x_t=b_t  \\
            & r_{t+1} \geq \alpha_{t+1}^j + \beta_{t+1}^j x_t & j=1,\ldots,k
\end{array}\right.
\]
\pula
\item Stage $t=1$
\[
\verm{\underline{z}^k}:=\left\{ \begin{array}{lll}
\displaystyle \min_{x_1\geq 0,r_2} &  c_1^\top x_1 + r_2\\
\mbox{s.t.} & A_1x_1=b_1 \\
           & r_{2} \geq \alpha_{2}^j + \beta_{2}^j x_1 & j=1,\ldots,k
\end{array}
\right.
\]

\end{itemize}
\end{frame}
%%%%%%%%%%%
\begin{frame}{Cutting-plane approximation}
\begin{itemize}
\item At stages $t=2,\ldots,T-1$
\[
\azul{\underline{Q_t}(x_{t-1}^k,\xi_{[t]})}:=
\left\{ \begin{array}{llll}
\displaystyle\min_{x_t\geq 0,r_{t+1}} & {c_t}^\top  x_t +r_{t+1} \\
\mbox{s.t.} & B_t x_{t-1}^{k} + A_t x_t=b_t  &&\verm{(\pi_t)}\\
            & r_{t+1} \geq \alpha_{t+1}^j + \beta_{t+1}^j x_t & j=1,\ldots,k & \verm{(\rho_j)}
\end{array}\right.
\]

\pula
\item Cuts ($t=T$)
\[
\alpha_T^k := \E_{|\xi_{T-1}}[b_T^\top \pi_T^k]\quad \mbox{and}\quad
\beta_T^k: =  -\E_{|\xi_{T-1}}[B_T^\top \pi_T^k]
\]
\pula
\item Cuts ($t=T-1,\ldots, 2$)
\[
\alpha_t^k := \E_{|\xi_{t-1}}[b_t^\top \pi_t^k + \sum_{j=1}^k \alpha_{t+1}^j \rho_k^j]\quad \mbox{and}\quad
\beta_t^k: =  -\E_{|\xi_{t-1}}[B_t^\top \pi_t^k]
\]
\end{itemize}
\end{frame}

%%%%
\begin{frame}{Nested decomposition - (nested L-shaped method)}
\begin{itemize}
\item  J.R. Birge (1985)
\end{itemize}
\begin{block}{It has two main steps:}
\begin{itemize}
\item \alert{Forward} that goes from $t=1$ up to $t=T$ solving subproblems to define policy $x_t^k(\xi_t)$
\begin{itemize}
\item In this step an upper bound $\overline{z}^k$ for the optimal value is determined
\end{itemize}

\pula
\item \alert{Backward} that comes from $t=T$ up to $t=1$ solving subproblems to compute linearizations that improve the cutting-plane approximation.

\begin{itemize}
\item In this step a lower bound $\underline{z}^k$ is obtained
\end{itemize}
\end{itemize}
\end{block}
\begin{block}{Stopping test}
\begin{itemize}
\item The Nested decomposition stops when
\[\overline{z}^k-\underline{z}^k \le \texttt{Tol}.\]
\item In this case $x_1^k$ is a $\texttt{Tol}$-solution to the T-SLP
\end{itemize}
\end{block}
\end{frame}
%%
%%
\begin{frame}{Forward and Backward steps}
\begin{center}
\includegraphics[width=7cm]{../Figs/EtapasProgRegr.pdf} {}
\end{center}
%{ \centerline{\includegraphics[width=7.5cm]{EtapasProgRegr.pdf}}}
\end{frame}
%%
\begin{frame}{Algorithm - Nested decomposition}
{stages $t=2,\ldots,T-1$
\[
\azul{\underline{Q_t}(x_{t-1}^k,\xi_{[t]})}:=
\left\{ \begin{array}{llll}
\displaystyle\min_{x_t\geq 0,r_{t+1}} & {c_t}^\top  x_t +r_{t+1} \\
\mbox{s.t.} & B_t x_{t-1}^{k} + A_t x_t=b_t  &&\verm{(\pi_t)}\\
            & r_{t+1} \geq \alpha_{t+1}^j + \beta_{t+1}^j x_t & j=1,\ldots,k & \verm{(\rho_j)}
\end{array}\right.
\]
}
{\scriptsize
\begin{itemize}
\item {\bf Step 0: inicialization}. Define $k=1$ and add the constraint $r_t = 0$ in all LPs $\underline{Q_t}$, $t=2,\ldots,T-1$
Compute $\underline{z}^1$ and let its solution be $x_1^{1}$
\pula
\item {\bf Step 1: forward}.
For t=2,\ldots,T, solve the LP $\underline{Q_t}$ to obtain $x_t^k:=x_t^k(\xi_{[t]})$. Define $\bar z^k := \E[\sum_{t=1}^T c_t^\top x_t^k]$
\pula
\item {\bf Step 2: backward}. Compute $\alpha_T^k$ and $\beta_T^k$.
Set $t=T$. Loop:
\begin{itemize}
\item While $t>2$
\item $t\gets t-1$
\item solve the LP $\underline{Q_t}(x_{t-1}^k,\xi_{[t]})$
\item Compute  $\alpha_t^k$ and $\beta_t^k$
\end{itemize}

Compute $\underline{z}^k$ and let its solution be $x_1^{k+1}$
\pula
\item {\bf Step 3: Stopping test}.  If $\bar z^k - \underline{z}^k\leq \epsilon$, stop. Otherwise set $k\gets k+1$ and \\ go back to Step 1
\end{itemize}
}
\end{frame}


%%%%%%%%%%%%%%%%%%%%%%%%%%%%%%%%%%%%
\begin{frame}{Nested decomposition - iterative process}
\begin{center}
\includegraphics[width=11cm]{../Figs/v1} {}
\end{center}
\begin{itemize}
\item $\min_{\underset{x_1\geq 0}{A_1x_1=b_1}} c_1^\top x_1 + \alert{\Q}_{2}(x_2,\xi_{[1]})$
\pula
\item $
\alert{\Q}_{t+1}(x_t,\xi_{[t]})=\E_{|\xi_{[t]}}[\verde{Q}_{t+1}(x_t,\xi_{[t+1]})]\; \mbox{ for }\; t=1,\ldots,T-1\,,
$
and  $\Q_{T+1}(x_T,\xi_{[T]}) =0$
\pula
%A fun��o de recurso m�dio $\Q$ usa as fun��es de recurso
\item $
\verde{Q}_{t}(x_{t-1},\xi_{[t]})=\min \; c_{t}^\top x_{t} +  \alert{\Q}_{t+1}(x_{t},\xi_{[t]})\;\mbox{ s.t. }\;\underset{x_{t}\geq 0}{B_{t}x_{t-1}+A_{t}x_{t}=b_{t}}
$
\end{itemize}
\end{frame}
%
\begin{frame}{Nested decomposition - iterative process}
\begin{center}
\includegraphics[width=11cm]{../Figs/v2} {}
\end{center}
\begin{itemize}
\item $\min_{\underset{x_1\geq 0}{A_1x_1=b_1}} c_1^\top x_1 + \azul{\check{\Q}}_{2}(x_2,\xi_{[1]})$
\pula
\item $
\azul{\check{\Q}}_{t+1}(x_t,\xi_{[t]})=\E_{|\xi_{[t]}}[\underline{{Q}_{t+1}}(x_t,\xi_{[t+1]})]\; \mbox{ for }\; t=1,\ldots,T-1\,,
$
e  $\check{\Q}_{T+1}(x_T,\xi_{[T]}) =0$
\pula
%A fun��o de recurso m�dio $\Q$ usa as fun��es de recurso
\item $
\underline{Q_{t}}(x_{t-1},\xi_{[t]})=\min \; c_{t}^\top x_{t} +  \azul{\Q}_{t+1}(x_{t},\xi_{[t]})\;\mbox{ s.t. }\;\underset{x_{t}\geq 0}{B_{t}x_{t-1}+A_{t}x_{t}=b_{t}}
$

\item \azul{Figures by Vincent Guigues}

\end{itemize}

\end{frame}
%
\begin{frame}{Nested decomposition - iterative process}
\begin{center}
\includegraphics[width=11cm]{../Figs/v3} {}
\doublebox{Forward step}
\end{center}

\end{frame}
%
\begin{frame}{Nested decomposition - iterative process}
\begin{center}
\includegraphics[width=11cm]{../Figs/v4} {}
\doublebox{Backward step}
\end{center}
\end{frame}
%
\begin{frame}{Nested decomposition - iterative process}
\begin{center}
\includegraphics[width=11cm]{../Figs/v5} {}
\doublebox{Backward step}
\end{center}
\end{frame}
%
\begin{frame}{Nested decomposition - iterative process}
\begin{center}
\includegraphics[width=11cm]{../Figs/v6} {}
\doublebox{Backward step}
\end{center}
\end{frame}
%
\begin{frame}{Nested decomposition - iterative process}
\begin{center}
\includegraphics[width=11cm]{../Figs/v7} {}
\doublebox{Backward step}
\end{center}
\end{frame}
%
\begin{frame}{Nested decomposition - iterative process}
\begin{center}
\includegraphics[width=11cm]{../Figs/v8} {}
\doublebox{Forward step}
\end{center}
\end{frame}
%
\begin{frame}{Nested decomposition - iterative process}
\begin{center}
\includegraphics[width=11cm]{../Figs/v9} {}
\doublebox{Forward step}
\end{center}
\end{frame}
%
\begin{frame}{Nested decomposition - iterative process}
\begin{center}
\includegraphics[width=11cm]{../Figs/v10} {}
\doublebox{Forward step}
\end{center}
\end{frame}
%
\begin{frame}{Nested decomposition - iterative process}
\begin{center}
\includegraphics[width=11cm]{../Figs/v11} {}
\doublebox{Forward step}
\end{center}
\end{frame}
%
\begin{frame}{Nested decomposition - iterative process}
\begin{center}
\includegraphics[width=11cm]{../Figs/v12} {}
\doublebox{Backward step}
\end{center}
\end{frame}
%
\begin{frame}{Nested decomposition - iterative process}
\begin{center}
\includegraphics[width=11cm]{../Figs/v13} {}
\doublebox{Backward step}
\end{center}
\end{frame}
%
\begin{frame}{Nested decomposition - iterative process}
\begin{center}
\includegraphics[width=11cm]{../Figs/v14} {}
\doublebox{Backward step}
\end{center}
\end{frame}
%
\begin{frame}{Nested decomposition - iterative process}
\begin{center}
\includegraphics[width=11cm]{../Figs/v15} {}
\doublebox{Backward step}
\end{center}
\end{frame}
%
\begin{frame}{Nested decomposition - iterative process}
\begin{center}
\includegraphics[width=11cm]{../Figs/v16} {}
\doublebox{Forward step}
\end{center}
\end{frame}
%
\begin{frame}{Nested decomposition - iterative process}
\begin{center}
\includegraphics[width=11cm]{../Figs/v17} {}
\doublebox{Forward step}
\end{center}
\end{frame}
%
\begin{frame}{Nested decomposition - iterative process}
\begin{center}
\includegraphics[width=11cm]{../Figs/v18} {}
\doublebox{Forward step}
\end{center}
\end{frame}
%
\begin{frame}{Nested decomposition - iterative process}
\begin{center}
\includegraphics[width=11cm]{../Figs/v19} {}
\doublebox{Forward step}
\end{center}
\end{frame}
%
\begin{frame}{Nested decomposition - iterative process}
\begin{center}
\includegraphics[width=11cm]{../Figs/v20} {}
\doublebox{Backward step}
\end{center}
\end{frame}
%
\begin{frame}{Nested decomposition - iterative process}
\begin{center}
\includegraphics[width=11cm]{../Figs/v21} {}
\doublebox{Backward step}
\end{center}
\end{frame}
%
\begin{frame}{Nested decomposition - iterative process}
\begin{center}
\includegraphics[width=11cm]{../Figs/v22} {}
\doublebox{Backward step}
\end{center}
\end{frame}
%
\begin{frame}{Nested decomposition - iterative process}
\begin{center}
\includegraphics[width=11cm]{../Figs/v23} {}
\doublebox{Backward step}
\end{center}
\begin{itemize}
\item Figures by Vincent Guigues.
\end{itemize}
\end{frame}

\begin{frame}{Convergence analysis}

\begin{block}{Assumptions}
\begin{itemize}
\item The set of nodes $\Omega_t$ has finitely many elements, $t=1,\ldots,T$
\pula
\item the problem has recourse relatively complete \azul{(for simplicity, only)}
\pula
\item the feasible set, in each stage $t=1,\ldots,T$, is compact

\end{itemize}
\end{block}
\pula
\begin{lemma}
$
\check \Q^k_t(x_{t-1},\xi_{[t-1]}) \leq  \Q_t(x_{t-1},\xi_{[t-1]}) \quad \forall\; x_{t-1} \; \mbox{ and }\; \forall t=2,\ldots,T
$
\end{lemma}
\pula

\begin{theorem}
The Nested Decomposition converges finitely to an optimal solution of the considered T-SLP
\end{theorem}
\end{frame}

\begin{frame}{Scenario tree}
  \tikzstyle{level 1}=[level distance=3cm, sibling distance=3.5cm]
  \tikzstyle{level 2}=[level distance=3cm, sibling distance=2cm]
  \tikzstyle{level 3}=[level distance=3cm, sibling distance=1cm]
  % Define styles for bags and leafs
  \tikzstyle{n1} = [circle, minimum width=10pt,fill, inner sep=0pt]

  % The sloped option gives rotated edge labels. Personally
  % I find sloped labels a bit difficult to read. Remove the sloped options
  % to get horizontal labels.
  \begin{tikzpicture}[grow=right, sloped]
    \node[n1] {}
    child {
      node[n1] {}
      child {
        node[n1, label=right:
        {}] {}
        child {
          node[n1, label=right:
          {$x_{3,8}$}] {}
          edge from parent
        }
        child {
          node[n1, label=right:
          {$x_{3,7}$}] {}
          edge from parent
        }
        edge from parent
      }
      child {
        node[n1, label=right:
        {}] {}
        child {
          node[n1, label=right:
          {$x_{3,6}$}] {}
          edge from parent
        }
        child {
          node[n1, label=right:
          {$x_{3,5}$}] {}
          edge from parent
        }
        edge from parent
      }
      edge from parent
    }
    child {
      node[n1] {}
      child {
        node[n1, label=right:
        {}] {}
        child {
          node[n1, label=right:
          {$x_{3,4}$}] {}
          edge from parent
        }
        child {
          node[n1, label=right:
          {$x_{3,3}$}] {}
          edge from parent
        }
        edge from parent
      }
      child {
        node[n1, label=right:
        {}] {}
        child {
          node[n1, label=right:
          {$x_{3,2}$}] {}
          edge from parent
        }
        child {
          node[n1, label=right:
          {$x_{3,1}$}] {}
          edge from parent
        }
        edge from parent
      }
      edge from parent
    };
  \end{tikzpicture}


  \begin{block}{Question}
    Can we compress the information?
  \end{block}

\end{frame}

\begin{frame}{Information structure}
  In multistage problems, decisions are sequential in nature
  \begin{equation*}
    x_0 \leadsto \xi_1 \leadsto x_1 \leadsto \xi_2 \leadsto \cdots \leadsto x_T
  \end{equation*}
  The sequence $\{\xi_t\}$ is a \emph{stochastic process}.

  \begin{block}{Non-anticipativity}
    We note $\xi_{[t]} := \{\xi_1, \cdots, \xi_{t-1}\}$ the information
    available up to time $t$.  \\
    The process $\{x_t\}_t$ is \emph{non-anticipative} if
    for all $t$, the values of $x_t$ depends only on the past information:
    \begin{equation*}
      x_t = x_t(\xi_{[t]}) \; .
    \end{equation*}
  \end{block}
\end{frame}


\begin{frame}{Information structure \#1: Stage-wise independence}
  \begin{block}{Stage-wise independence}
    Assume uncertainties $\xi_1, \cdots, \xi_T$ are time-step independent: \\
    for all $t$, the random variable $\xi_t$ is stochastically independent from $\xi_{[t-1]}$.
  \end{block}

  \vspace{2cm}

  \tikzstyle{n1} = [circle, minimum width=10pt,fill, inner sep=0pt]

  % The sloped option gives rotated edge labels. Personally
  % I find sloped labels a bit difficult to read. Remove the sloped options
  % to get horizontal labels.
  \begin{tikzpicture}[grow=right, sloped]
    \node[n1, label=above:{$x_0$}] (N1) at (0, 0) {};
    \node[n1, label=above:{$x_1$}] (N2) at (3, 0) {};
    \node[n1, label=above:{$x_2$}] (N3) at (6, 0) {};
    \node[n1, label=above:{$x_3$}] (N4) at (9, 0) {};
    \draw (N1) -- (N2);
    \draw (N2) -- (N3);
    \draw (N3) -- (N4);
  \end{tikzpicture}

  \vspace{2cm}

  The tree reduces to a line: \\ we will prove that $x_4$ depends only on the values in $x_3$,
  not on $x_2, x_1$

\end{frame}



%%%%%
\begin{frame}{Dynamic programming formulation}
  \begin{block}{Proposition}
    Under stage-wise independence, the value functions $\{Q_t(\cdot, \xi)\}$
    satisfy \\ the Dynamic Programming backward recursions:
    setting $V_{T+1}(x_{T}) := 0$, we have for $t=T, \cdots, 1$,
    \begin{equation*}
      \left\{
      \begin{aligned}
      & Q_t(x_{t-1}, \xi) = \min_{x_t \geq 0 } \big[ c_t(\xi)^\top x_t + V_{t+1}(x_{t}) \big] \quad \text{s.t.} \quad B_t(\xi) x_{t-1} + A_t(\xi) x_t = b_t(\xi) \\
      & V_t(x_{t-1}) = \mathbb{E}_{\xi} \big[ Q_t(x_{t-1}, \xi) \big]
      \end{aligned}
      \right.
    \end{equation*}
  \end{block}

  \begin{block}{Theorem}
    A feasible policy $\{\overline{x}_t \}_t$ is \emph{optimal}
    if and only if, for all $t=1, \cdots, T$,
    \begin{equation*}
      \overline{x}_t(\xi_{[t]}) \in \argmin_{x_t \geq 0 } \; c_t(\xi_t)^\top x_t + V_{t+1}(x_t) \quad \text{s.t.} \quad B_t(\xi_t) x_{t-1} + A_t(\xi_t) x_t = b_t(\xi_t)
    \end{equation*}
    A direct corollary is that $\overline{x}_t(\xi_{[t]})$ depends only
    on the values of $x_{t-1}$ and of $\xi_t$.
  \end{block}

\end{frame}

%%%%
\begin{frame}{Optimality conditions: sensitivity propagate backward}

\begin{itemize}
\item Let the extended real-valued function $f_t(x_t,\xi) = \left\{\begin{aligned}
    c_t(\xi)^\top x_t & \quad \text{if } \quad x_t \geq 0  \\
    +\infty & \quad \text{otherwise}
\end{aligned}\right.$
\item Let the Lagrangian
  \begin{equation*}
    L(x_t, \pi_t) := f_t(x_t, \xi) + V_{t+1}(x_t) + \pi_t (b_t - B_t x_{t-1} - A_t x_t)
  \end{equation*}

% \pula
% \item $\bar \pi _t(\xi_{[t]}) \in \D(\bar x_{t-1}(\xi_{[t-1]}),\xi_{[t]})$
% \pula
% \item $\D(\bar x_{t-1}(\xi_{[t-1]}),\xi_{[t]})$ is the set of Lagrange multipliers of
% \[Q_t(x_{t-1},\xi_{[t]}):= \min_{\underset{x_t\ge 0}{B_tx_{t-1} + A_tx_t=b_t}} c_t^\top x_t +
% \Q_{t+1}(x_t,\xi_{[t]})\]
\end{itemize}

\begin{theorem}
Under some assumptions (e.g. finitely many scenarios and polyhedral $f_t$).
A feasible policy $\bar x_t(\xi_{[t]})$ is optimal iff there exists measurable $\bar \pi_t(\xi_{[t]})$, $t=1,\ldots,T$, such that
\[
  \verm{0\in \partial f_t(\bar x_t(\xi_{[t]}),\xi_t)  + \partial V_{t+1}(\bar x_t(\xi_{[t]})) - A_t^\top \bar \pi_t(\xi_{[t]})}
% - \E_{|\xi_{[t]}}[B_{t+1}^\top \bar \pi_{t+1}(\xi_{[t+1]})
\]
and
\begin{equation*}
  \verm{ \partial V_t(x_{t-1}) = \mathbb{E}_\xi \big[ - B_t^\top \bar \pi_t(\xi_{[t]}) \big]}
\end{equation*}
 for a.e. $\xi_{[t]}$ and $t=1,\ldots,T$
\end{theorem}
\end{frame}

\begin{frame}{Solution algorithm by discretization}
\RestyleAlgo{ruled}
  \begin{algorithm}[H]
    \caption{Stochastic Dynamic Programming (SDP)}
    \KwData{Set $V_{T+1} = 0$}
    Find a discretization $\mathbb{X}_d$ of feasible space $\mathcal{X}$;

    \For{$t=T, \cdots, 1$}{
      \For{$x_{t-1} \in \mathbb{X}_d$}{
        \For{$\xi_t \in \{\xi_t^1, \cdots, \xi_t^N \}$}{
          Solve $Q_t(x_{t-1}, \xi_t)$ using linear programming;
        }
        Set $V_{t}(x_{t-1}) = \sum_{i=1}^N p_i Q_t(x_{t-1}, \xi_t^i)$;
      }
    }
  \end{algorithm}

  \begin{block}{Complexity}
    Algorithm SDP solves a total number of (small) LP:
    \begin{equation*}
      \mathcal{O}(|\mathbb{X}_d| \times N \times T)
    \end{equation*}
  \end{block}
  Algorithm is useful only if dimension of the state is small
  (\emph{curse of dimensionality})
\end{frame}


%%%%

\begin{frame}{Information structure \#2: Markov chain}
  \begin{block}{Markov-chain lattice}
    The process $\{\xi_t\}_t$ is Markovian if for all $t$, the conditional
    distribution of $\xi_t$ given $\xi_{[t-1]}$ depends only on $\xi_{t-1}$.
  \end{block}

  \vspace{2cm}

  \tikzstyle{n1} = [circle, minimum width=10pt,fill, inner sep=0pt]

  % The sloped option gives rotated edge labels. Personally
  % I find sloped labels a bit difficult to read. Remove the sloped options
  % to get horizontal labels.
  \begin{tikzpicture}[grow=right, sloped]
    \node[n1, label=above:{$\xi_1$}] (N1) at (0, 1) {};
    \node[n1, label=above:{$\xi_{2,1}$}] (N21) at (3, 0) {};
    \node[n1, label=above:{$\xi_{2,2}$}] (N22) at (3, 2) {};
    \node[n1, label=above:{$\xi_{3,1}$}] (N31) at (6, 0) {};
    \node[n1, label=above:{$\xi_{3,2}$}] (N32) at (6, 2) {};
    \node[n1, label=above:{$\xi_{4,1}$}] (N41) at (9, 0) {};
    \node[n1, label=above:{$\xi_{4,2}$}] (N42) at (9, 2) {};
    \draw (N1) -- (N21);
    \draw (N1) -- (N22);
    \draw (N21) -- (N31);
    \draw (N21) -- (N32);
    \draw (N22) -- (N31);
    \draw (N22) -- (N32);
    \draw (N31) -- (N41);
    \draw (N31) -- (N42);
    \draw (N32) -- (N41);
    \draw (N32) -- (N42);
  \end{tikzpicture}

\end{frame}




%---------------------------------------------------------------------
\section{SP with recourse: Sample Average Approximation - SAA}
\usebackgroundtemplate{\includegraphics[width=\paperwidth]{../Figs/coverlight}}
\begin{frame}[noframenumbering,plain]{ }
\begin{block}{\Large SP with recourse}
 Sample Average Approximation - SAA
\end{block}
\end{frame}
\usebackgroundtemplate{\includegraphics[width=\paperwidth]{../Figs/white}}
\begin{frame}{Stochastic programs with recourse} 
Let's consider SP of the form
\[
\min_{x \in X}\, \E[f(x,\omega)]\,
\]
\pula
with
\begin{itemize}
\item $f:\Re^n\times \Omega \to \Re$ is convex on $x$ (decision variable)
\item $X \subset \R^n$ is a deterministic set (e.g. a fixed polyhedron)
\item $\omega$ is a random vector and $(\Omega,\mathcal{F},P)$ is its probability space
\item $\E[\cdot]$ is the expected value w.r.t. the probability measure $P$
\end{itemize}
\pula
For instance, in a two-stage stochastic linear framework, we have
\[
f(x,\omega) = c^\top x + Q(x,\omega)
\]
with
\[Q(\verm{x},\azul{\omega}):=\left\{ \begin{array}{rll}
\min & q(\azul{\omega})^\top y &\\
\mbox{s.t.} & W(\azul{\omega}) y   =h(\azul{\omega})-T(\azul{\omega})\verm{x}&\\ 
& y  \geq 0  &
\end{array} \right.\]
\end{frame}


%\begin{frame}{Example} 
%{
%\[\hspace{-1cm}(P)\quad \left\{ \begin{array}{cl}\displaystyle \min_{x,y(\omega)} & 2x_{1} + 3x_{2} +\E[ 7y_1(\omega) + 12y_2(\omega)]\\
%& \begin{array}{rrrll}
%x_{1} & + x_{2} &  &  \leq 100&\\
%2x_{1}  &+ 6x_{2} &+y_1(\omega) &  \geq h_1(\omega)\\
%3x_{1}  &+ 3x_{2} &  +y_2(\omega) & \geq h_2(\omega)\\
% & x_{1} \,, x_{2}\geq0\,,& y_1(\omega) \,, y_2(\omega)  & \geq 0  
%\end{array}
%\end{array}\right.\]
% }
%\verm{The studied example of oil/gasoline management fits this formulation}
% \pula
% 
%Define $f(x,\verm{\omega}) = 2x_{1} + 3x_{2} + Q(x,\verm{\omega})$, with 
%\[Q(x,\verm{\omega}):=\left\{
%\begin{array}{lllll}
%\displaystyle \min_{y\geq 0} &7y_1 + 12y_2\\
%\mbox{s.t.}
%&y_1   \geq h_1(\verm{\omega}) -(2x_{1}  + 6x_{2} )\\
%&  y_2 \geq h_2(\verm{\omega})-(3x_{1}  + 3x_{2} )
%\end{array}\right.
%\]
%
%which is equivalent to 
%\[\azul{Q(x,\omega)=
%7[h_1(\omega) -(2x_{1}  + 6x_{2} )]^+ + 12[ h_2(\omega)-(3x_{1}  + 3x_{2} )]^+
%}
%\]
%Therefore, (P) can be written as
%\[
%\min_{x} \E[f(x,\omega)]\quad \mbox{s.t.}\quad x \in X:=\{x_1,x_2\geq0, \; x_1+x_2 \leq 100\}
%\]
%\pula
%
%\azul{W.l.o.g. we stick with the generic formulation of SP with recourse:
%\[
%\min_{x \in X}\, \E[f(x,\omega)]
%\]
%}
%\end{frame}


\begin{frame}{Representation of the uncertainties}
\begin{block}{Continuous probability distribution}
\begin{itemize}
\item Sample space  $\Omega$ contains infinitely many elements
\[
\min_{x \in X}\, \E[f(x,\omega)]
\]

\item  For computational reasons, it is necessary to  consider finitely many scenarios $\omega^i \in \Omega$, with associated probability $p_i>0$
\pula

\item Resulting problem
\[ 
\min_{x \in X}\, f^N(x)\quad \mbox{with}\quad f^N(x):=\sum_{i=1}^{N}p_if(x,\omega^i)
\]
\end{itemize}
\end{block}
\begin{block}{Sample Average Approximation - SAA}
\[
\min_{x \in X}\, \frac{1}{N}\sum_{i=1}^{N}f(x,\omega^i)
\]
\end{block}
 \end{frame}
 
 
 
\begin{frame}
\begin{block}{How to proceed when we do not know $P$?}
\begin{itemize}
\item In many applications the probability distribution is not precisely known
\pula

\item In these cases, $ P $ is estimated by using the historical of the stochastic vector
\pula

\item Scenario generation can be done via Monte Carlo simulation (it is advisable not to use
the historical as scenarios)
\end{itemize}
\pula
\end{block}
\begin{block}{Representation of the uncertainties}
\begin{center}
\includegraphics[width=5cm]{../Figs/formatos.pdf} {}
\end{center}
\end{block} 
\end{frame}





\begin{frame}{The newsvendor problem}

\begin{itemize}


\item A newsvendor buys newspaper  by the morning at price \azul{$c$} and sells them along the day at price \azul{$r$}
\pula

\item Unsold newspaper are sent to be recycled. The value earned by every recycled newspaper is \azul{$s$}
\pula

\item The newsvendor wishes to maximize its expected income:
\[
\min_{x\geq 0} \E[f(x,\omega)],
\]
where
\[
\begin{array}{lll}
f(x,\omega) &= -[-cx + r\min\{x,\omega\} +  s(x - \min\{x,\omega\})]\\
&=
-[(s-c)x + (r-s)\min\{x,\omega\}]
\end{array}
\]
\end{itemize}

 \end{frame} 





%\begin{frame}{Sampling}

%\begin{itemize}

%\item For every given $x$ we could think of approximating $f(x)=\E[f(x,\omega)]$  by a sample average
%\pula


%\item I.e., for every given $x \in X$ we can sample $\{\omega_x^1,\ldots, \omega_x^N\}$ and approximate $f(x)$
%by
%\[
%f^N(x) = \frac{1}{N}\sum_{i=1}^N f(x,\omega_x^i)
%\]
%\pula
%\pause
%\item Generating a different sample for each $x$ is useless:
%\end{itemize}
%\begin{center}
%\includegraphics[width=6cm]{Figs/jornaleiro1.png} {}
%\end{center}
%\end{frame} 
 
 

\begin{frame}{Sample Average Approximation - SAA}

\begin{itemize}
\item The main idea of the \azul{Sample Average Approximation} - SAA - approach is to use the same sample for all $x \in X$
\pula

\item I.e., we draw a sample $\{\omega^1,\ldots, \omega^N\}$ and approximate $f(x)$ by
\[
f^N(x) = \frac{1}{N}\sum_{i=1}^N f(x,\omega^i)
\]
regardless the given point $x$
\end{itemize}
\begin{center}
\includegraphics[width=6.5cm]{../Figs/jornaleiro2.png} {}
\end{center}
 \end{frame} 




\begin{frame}{Sample Average Approximation - SAA}

\begin{itemize}

\item The approximation of $f^N$ of $f$ is quite close to $ f $
\pula

\item This suggests replacing the original problem $ \min_{x \in X} f (x) $ by
\[
\min_{x \in X} f^N (x) 
\]
which can be solved by deterministic methods (L-Shaped, Nested Decomposition, Bundle Method, etc.)
\end{itemize}
\begin{block}{Questions}
\begin{itemize}
\item Does the SAA approach always work regardless the function $f(x,\omega)$?
\pula

\item What is a good size $N$ of the sample to be considered?
\pula

\item What can we say about the quality of the SAA solution?
\end{itemize}
\end{block}

 \end{frame} 
 
 
 

\begin{frame}{Asymptotic properties}

Let 
\begin{itemize}
\item $\hat x^N$ be a solution of the SAA problem
\item $\hat S^N$ be the solution set of the SAA problem
\item $\hat f^N$ be the optimal value of the SAA problem
\end{itemize}
and
\begin{itemize}
\item $ x^*$ be a solution of the true problem
\item $S^*$ be the solution set of the true problem
\item $f^*$ be the optimal value of the true problem
\end{itemize}


\begin{block}{Questions}
\begin{itemize}
\item $\lim_{N\to \infty} \hat x^N = x^*$?
\pula 
\item $\lim_{N\to \infty} dist(S^N,S^*) = 0$?
\pula
\item $\lim_{N\to \infty} \hat f^N = f^*$?

\end{itemize}
\end{block}
 \end{frame} 
 
 
 
\begin{frame}{Asymptotic properties}
\begin{itemize}
\item Firstly, let's try to answer these questions by considering the newsvendor problem
\pula

\item Suppose the demand for newspaper follows an Exponential probability distribution
\[
\omega \sim {\tt Exponencial}(10),
\quad \mathbb{P}[\omega \leq x]= 1 - e^{-10x}\quad \mbox{(if $x\geq 0$)}
\]
\end{itemize}
\begin{center}
\includegraphics[width=9cm]{../Figs/jornaleiro3.png} {}
\end{center}
\end{frame} 





\begin{frame}{Asymptotic properties}

\begin{itemize}

\item The table presents the values of $\hat x^N$ and $\hat f^N$ for some different sample size $N$

\begin{center}
\begin{tabular}{|c|c|c|c|c|c|}
\hline 
N & 10 & 30 & 90 & 270 & $\infty$ \\ 
\hline 
$\hat x^N$ & 1.46 & 1.44 & 1.54 & 2.02 & 2.23 \\ 
\hline 
$\hat f^N$ & -1.11 & -0.84 & -0.98 & -1.06 & -1.07 \\ 
\hline 
\end{tabular} 
\end{center}
\item It seems that $f^N$ approximates well $f^*$ when $N$ increases
\pula

\item Notice that $\hat x^N \to x^*$ and $\hat f^N \to f^*$
\end{itemize}
 \end{frame} 
 
 

\begin{frame}{Asymptotic properties}

\begin{itemize}
\item Suppose now that demand for newspaper follows a discrete uniform probability distribution on the set
\[
\{1,3,\ldots, 10\}\,
\]
\end{itemize}
\begin{center}
\includegraphics[width=9cm]{../Figs/jornaleiro4.png} {}
\end{center}
\end{frame} 
 
 
\begin{frame}{Asymptotic properties}

\begin{itemize}

\item The table presents the values of $\hat x^N$ and $\hat f^N$ for some different sample size $N$


\begin{center}
\begin{tabular}{|c|c|c|c|c|c|}
\hline 
N & 10 & 30 & 90 & 270 & $\infty$ \\ 
\hline 
$\hat x^N$ & 2 & 3 & 3 & 2 & [2,3] \\ 
\hline 
$\hat f^N$ & -2.00 & -2.50 & -1.67 & -1.35 & -1.50 \\ 
\hline 
\end{tabular} 
\end{center}

\item Again, it seems that  $f^N$ approximates well $f^*$ when $N$ increases
\pula

\item Notice that  $\hat f^N \to f^*$, but $\hat x^N$ does not seem to converge 
\pula
\item $\hat x^N$ is oscillating between two optimal solutions of the problem
\pula

\item What can we conclude?
\end{itemize}

 \end{frame} 

\begin{frame}{Convergence results}



\begin{itemize}
\item In both cases (continuous and discrete) the function  $f^N$ converges uniformly to $f$
\pula

\item Uniform convergence occurs, for instance, when $f$ is continuous 
\pula

\item When continuous convergence is observed, we have the following results
\end{itemize}

\begin{theorem}

\begin{itemize}
\item $\lim_{N \to \infty} \hat f^N = f^*$ with probability 1 (w.p.1)
\pula
\item Suppose there exists a compact set $C$ such that
\begin{itemize}
\item $\emptyset\neq S^* \subset  C$ and $\emptyset\neq S^N \subset  C$ (w.p.1) for $N$ large enough
\item the objective function is continuous and finite-valued on $C$ 
\end{itemize}
\pula
\verm{Then $\lim_{N \to \infty} dist(S^N,S^*) = 0$ w.p.1}
\end{itemize}
\end{theorem}
 \end{frame} 


\begin{frame}{What does ``convergence w.p.1" mean?}

\begin{itemize}
\item  Each function  $f^N$ is constructed with a single sample
\pula
\item Regardless the sample,  convergence results hold provided that $N \to \infty$ 
\end{itemize}
\pula
Let's repeat the same experiment for $N=270$ several times:


\begin{center}
\includegraphics[width=12cm]{../Figs/jornaleiro5.png} {}
{\small
(a) Exponential distribution \quad \quad (b) Discrete uniform distribution}
\end{center}
 \end{frame} 
 
 
 
\begin{frame}{What does ``convergence w.p.1" mean?}


\begin{itemize}
\item For some samples with $N=270$ the approximation is quite good. However, for other samples the approximation is poor
\pula

\item Why don't we have convergence for every sample?
\pula
\pause

\item \verm{The theorem only ensures convergence when $N \to \infty$...}
\end{itemize}

\pula
\azul{Given a sample of size of $ N $, we solve the SAA problem $\min_{x \in X} f^N (x) $
\begin{center}
\shadowbox{How do we know if we have a ``good'' or ``bad '' sample?}
\end{center}
}

\pause
\begin{itemize}
\item The answer is: \verm{we do not know for sure}. However, we may employ simulation and statistical tools to access quality
\end{itemize}

 \end{frame} 


\begin{frame}{A result on the sample size}

\[
 \min_{x \in X} f(x)=\E[f(x,\omega)] \quad \quad (SAA) \quad \min_{x \in X} f^N(x)=\frac{\sum_{i=1}^N f(x,\omega^i)}{N}
\]


\begin{theorem}
Suppose the true stochastic program has a unique solution $x^*$, $X$ is a compact set and $f(\cdot, \omega)$ is strongly convex. 
Then, for all $\epsilon >0$ there exist constants $\beta(\epsilon)>0$ and $C>0$ such that
\[
\mathbb{P}[\|\hat x^N- x^*\|>\epsilon] \leq  C e^{-N\,\beta(\epsilon)}
\]

\pause 

\azul{The theorem ensures the existence of such constants, but not their values}
\pula

\verm{We need to perform simulation...}
\end{theorem}

 \end{frame} 
  
  
  
\begin{frame}{Simulation}

\begin{itemize}
\item Given a sample $\{\omega^1,\ldots,\omega^N\}$, define
\[
\hat x^N \in \arg \min_{x \in X} f^N(x), \quad \mbox{and}\quad f^N(x)=\frac{\sum_{i=1}^N f(x,\omega^i)}{N}\quad (SAA)
\]
\pula
\item In order to assess the quality of $\hat x^N$ it is mandatory to generate a  larger sample $\{\tilde \omega^1,\ldots,\tilde \omega^{N'}\}$ independent of $\{\omega^1,\ldots,\omega^N\}$ and evaluate the costs
\[
f(\hat x^N, \tilde \omega^j), \quad j=1,\ldots N' \;\; (>> N)
\]
(\azul{Evaluating the function is easier than solving the SAA problem})
\end{itemize}

 \end{frame} 



\begin{frame}{Simulation}
\[
\hat x^N \in \arg \min_{x \in X} f^N(x), \quad \quad f^N(x)=\frac{\sum_{i=1}^N f(x,\omega^i)}{N}\quad (SAA)
\]

\begin{itemize}
\item In order to infer if the sample size $N$  is satisfactory we may compare
\pula
\begin{itemize}
\item $f^N(\hat x^N)$ with $f^{N'}(\hat x^N)$   (average of the individual costs $f(\hat x^N, \tilde \omega^j)$)
\pula

\item Empirical distribution of the individual costs  $f(\hat x^N, \tilde \omega^j),\;\;j=1,\ldots,N'$ and $f(\hat x^N,\omega^i),\;\;i=1,\ldots,N$ (KS-test)
\end{itemize}
\end{itemize}

\begin{center}
\includegraphics[width=5cm]{../Figs/KSteste.png} {}
\end{center}
 \end{frame} 
 
\begin{frame}{Simulation}

Another idea widely used in practice is:
\begin {itemize}
\item Given $ N $ and $ M $, generate $ M $  samples of size $ N $ and solve $ M $ problems $ SAA $
\pula

\item Compare the most important variables of the $ M $  SAA solutions $ \hat x^N_i$, $ i = 1, \ldots, M$
\pula

\item Evaluate the SAA solutions using a larger sample $\{ \tilde \omega^1, \ldots, \tilde \omega^{N'} \}$
\pula

\item Compare the empirical cost distributions
\end{itemize}
\pula

\azul{If there is a certain ``adherence" among the results, the size $ N $ can be considered satisfactory. Otherwise, it is suggested to increase $ N $}

\pula
\verde{Importance of simulation}
\begin{itemize}
\item It allows  us to analyze the quality of solution obtained with the stochastic model
\pula

\item it allows us to estimate an appropriated sample size
\end{itemize}
\end{frame} 

%%%%%%%%%%%%%%%

\begin{frame}{Computing confidence intervals}
\verm{Let $x$ be fixed }(for instance, $x=x^N$ the solution of the SAA model)

\pula

\azul{The strong law of large numbers} ensures, for $N$ large enough
\[
f^N(x):=\frac{1}{N}\sum_{i=1}^N f(x,\omega^i) \approx \int_{\Omega}f(x,\omega)dP(\omega)=\E[f(x,\omega)]:=f(x)
\]
w.p.1, provided that $\{\omega^1,\ldots, \omega^N\}$ is a iid sample of $\omega$ and $\E[\omega]$ is finite (no assumption on the distribution of $\omega$ is required!)
\pula

Thus, we can use $f^N(x)$ as an approximation of \[f(x)=\E[f(x,\omega)]\]

\begin{block}{Difficulties}
\begin{itemize}
\item $f^N(x)$ is a random variable itself: it depends on the sample $\{\omega^1,\ldots, \omega^N\}$
\pula

\item Sometimes $f^N$ can be an accurate approximation of $f$, sometimes not
\end{itemize}
\end{block}
 \end{frame} 

\begin{frame}{The Central Limit Theorem }
 As $f^N$ is a random variable, it makes sense to compute its mean and variance. Let $x$ be a given point:
\[
\E[f^N(x)] = \E[\frac{1}{N}\sum_{i=1}^N f(x,\omega^i)]= \frac{1}{N}\sum_{i=1}^N \E[f(x,\omega^i)] =f(x)
\]
\azul{Therefore, $f^N(x)$ is an unbiased estimator of $f(x)$}

\pula
\[
\begin{array}{ll}
\var[f^N(x)]&= \var[\frac{1}{N}\sum_{i=1}^N f(x,\omega^i)]=  \frac{1}{N^2}\sum_{i=1}^N \var[f(x,\omega^i)]= \frac{1}{N} \var[f(x,\omega)]
\end{array}
\]
\verm{ Thus, the variance of $f^N(x)$ vanishes when $N$ goes to infinity} (if $\sigma^2=\var[f(x,\omega)]$ is finite, of course)


\begin{block}{The Central Limit Theorem - CLT}
\[
\frac{\sqrt{N}[f^N(x) - f(x)]}{\sigma}\approx \mathcal{N}(0,1)
\]
\end{block}
 \end{frame} 
 

\begin{frame}{The Central Limit Theorem }
\[
\frac{\sqrt{N}[f^N(x) - f(x)]}{\sigma}\approx \mathcal{N}(0,1)
\]
\pula 

The CLT ensures that, for an arbitrary $x \in {\tt dom}\, (f)$, $\sqrt{N}[f^N(x) - f(x)]$
converges in distribution to normal distribution with zero mean and variance equal to $\sigma^2=\var[f(x,\omega)]$ (provided that this variance is finite)
\pula

Consequently $f^N(x)$ converges to $f(x)$ at a (stochastic) rate of $\mathcal{O}_1(N^{-1/2})$
\pula

In other words, in order to estimate $f(x)$ by its sample average with an
accuracy $\epsilon>0$ one needs a sample of size $N = \mathcal{O}_2(\epsilon^{-2})$
\pula 

Although various techniques, e.g., variance reduction techniques and quasi-Monte Carlo
methods, were developed in simulation literature in order to enhance the
accuracy of such estimates,  it is basically impossible to evaluate multivariate integrals with
a high precision
\pula

As a conclusion, when solving a SAA problem, keep in mind that you're solving only an (possibly rough) approximation of the ``true" stochastic program

\pula
However, there are manners to estimate how good or how bad is the SAA solution...

 \end{frame} 

\begin{frame}{Computing the error of the estimated value}
{Confidence interval}

The CLT ensures that
\[
P\left[f^N(x) - 1.96\frac{\sigma}{\sqrt{N}} \leq f(x) \leq f^N(x) + 1.96\frac{\sigma}{\sqrt{N}}\right]= 0.95
\]
\pula

In practical terms, the above expression  means that for each 100 samples $\{\omega^1,\ldots,\omega^N\}$, the true value $f(x)$ is contained in 95 intervals
\[
\left[f^N(x) - 1.96\frac{\sigma}{\sqrt{N}} ,\, f^N(x) + 1.96\frac{\sigma}{\sqrt{N}}\right]
\]
\pula

The variance  $\sigma^2$ is generally unknown, but it can be estimated by
\[
S^2 = \frac{\sum_{i=1}^N [f(x,\omega^i) - f^N(x)]^2}{N-1}
\]
 \end{frame} 


\begin{frame}{ }
The true stochastic optimization problem can be solved by
the SAA approach in a reasonable time with a reasonable accuracy provided
that the following conditions are satisfied:
\begin{itemize}
\item  (i) the required sample size is
manageable
\item  (ii) it is possible to solve the constructed SAA problem with
a reasonable efficiency
\end{itemize} 
\pula 

From this point of view the number of scenarios of
the true problem (i.e., cardinality of the support $\Omega$ of the distribution of $\omega$)
is irrelevant and can be infinite
\pula

Condition (ii) holds in the case of two
stage linear stochastic programming with recourse
%\pula

%Condition (i) holds if for every feasible $x$ variability of $f(x,\omega)$ is ``not too large"
%(in particular,
%$f(x,\omega)$ should be finite valued for a.e. realization of $\omega$
\end{frame}
  
  
\begin{frame} {Some conclusions} 
\begin{itemize}

\item Stochastic optimization decisions are in general balanced: protect against ``bad" scenarios
\pula

%\item Solutions depend (strongly) on the considered scenarios
%\pula

\item The complexity of the optimization problem grows with the number of
scenarios
\pula

\item A SAA problem is frequently used to approximate a stochastic program
\pula

\item Once a SAA solution is determined, it is crucial to evaluate the quality of the output through simulations
\end{itemize}

\end{frame}
%=======================================================


%---------------------------------------------------------------------
\section{SP with recourse: Estimating optimality gap}

\usebackgroundtemplate{\includegraphics[width=\paperwidth]{../Figs/coverlight}}
\begin{frame}[noframenumbering,plain]{ }
\begin{block}{\Large Estimating optimality gap}
\end{block}
\end{frame}
\usebackgroundtemplate{\includegraphics[width=\paperwidth]{../Figs/white}}
\begin{frame}{Stochastic programs with recourse} 
Let's consider SP of the form
\[
\min_{x \in X}\, \E[f(x,\omega)] 
\]
where
\begin{itemize}
\item $f:\Re^n\times \Omega \to \Re$ is convex on $x$ (decision variable)
\item $X \subset \R^n$ is a deterministic set (e.g. a fixed polyhedron)
\item $\omega$ is a random vector and $(\Omega,\mathcal{F},P)$ is its probability space
\item $\E[\cdot]$ is the expected value w.r.t. the probability measure $P$
\end{itemize}
\end{frame}


\begin{frame}{Representation of the uncertainties}
\begin{block}{Continuous probability distribution}
\begin{itemize}
\item Sample space  $\Omega$ contains infinitely many elements
\[
\min_{x \in X}\, \E[f(x,\omega)]
\]

\item  For computational reasons, it is necessary to  consider finitely many scenarios $\omega^i \in \Omega$, with associated probability $p_i>0$
\pula

\item Resulting problem
\[ 
\min_{x \in X}\, f^N(x)\quad \mbox{with}\quad f^N(x):=\sum_{i=1}^{N}p_if(x,\omega^i)
\]
\end{itemize}
\end{block}
\begin{block}{Sample Average Approximation - SAA}
\[
\min_{x \in X}\, \frac{1}{N}\sum_{i=1}^{N}f(x,\omega^i)
\]
\end{block}
 \end{frame}
 
 
 
 
\begin{frame}
\begin{block}{How to proceed when we do not know $P$?}
\begin{itemize}
\item In many applications the probability distribution is not precisely known
\pula

\item In these cases, $ P $ is estimated by using the historical of the stochastic vector
\pula

\item Scenario generation can be done via Monte Carlo simulation (it is advisable not to use
the historical as scenarios)
\end{itemize}
\pula
\end{block}
\begin{block}{Representation of the uncertainties}
\begin{center}
\includegraphics[width=5cm]{../Figs/formatos.pdf} {}
\end{center}
\end{block} 
\end{frame}










 
 

 
 
\begin{frame}{Evaluating a candidate solution }

The point $\hat x^N$ solution  of
\[
(SAA)\quad \hat f^N=\min_{x \in X} f^N(x)\quad \mbox{with}\quad f^N(x)=\frac{\sum_{i=1}^N f(x,\omega^i)}{N}
\]
is a candidate solution to the ``true" problem
\[
f^*=\min_{x \in X} f(x)\quad \mbox{with}\quad f(x)=\E[f(x,\omega)] 
\]
\pula

As the feasible set of both problems are the same, we get \azul{${f}(\hat x^N)\geq f^*$}

\pula
We thus have the following \verm{unknown} optimality gap
\[
{\tt gap} (\hat x)=f(\hat x^N)- f^*\geq 0
\]

\azul{In what follows we are going to find an estimator $\hat{{\tt gap}}(\hat x^N) $ for ${\tt gap} (\hat x^N)$}
 \end{frame}
 
 
\begin{frame}{An upper bound (confidential interval) for ${\tt gap}(\hat x^N)$}
\begin{itemize}
\item Generate randomly, from the probability distribution of $\omega$, a (iid) sample with $N$ scenarios 
\pula
\item Obtain  $\hat x^N$ a solution of the resulting SAA problem
\pula
\item Generate randomly, from the probability distribution of $\omega$, another (iid) sample with $N'>>N$ scenarios 
\pula
\item Compute $f^{N'}(\hat x^N)= \frac{1}{N'}\sum_{i=1}^{N'} f(\hat x^N,\omega^i)$
\pula
\item Compute the variance of $f^{N'}(\hat x)$
\[
\hat \sigma_{N'}^2(\hat x^N):=\frac{1}{N'(N'-1)}\sum_{i=1}^{N'} [f(\hat x^N,\omega^i)-f^{N'}(\hat x^N)]^2
\]
\item Compute the upper bound for the  $100(1-\alpha)$-confidential interval of $f(\hat x^N)$:
\azul{
\[
U_{N}'(\hat x^N):= f^{N'}(\hat x^N) + z_{\alpha}\hat \sigma_{N'}(\hat x^N)\,,
\]
}
where $z_\alpha=\Phi^{-1}(1-\alpha)$ and $\Phi(z)$ is standard normal distribution. Ex: if $\alpha =5\%$, then $z_\alpha\approx 1.64$.
\end{itemize}
\end{frame} 

\begin{frame}{ }
\begin{itemize}
\item Before defining a lower bound for the confidential interval of  ${\tt gap} (\hat x^N)$, notice that, for every given $x \in X$
\[
f(x)=\E[f^N(x)]\geq \E[\min_{y \in X} f^N(y)]=\E[\hat f^N]
\] 
(\azul{because  $f^N(x)$ is an unbiased estimator of $f(x)$})
\pula

\item Hence, we get the following useful inequality 
\[
f^*\geq \E[\hat f^N]
\]
\pula
\item \azul{We can estimate $\E[\hat f^N]$ by solving several SAA problems}
\end{itemize}
\end{frame}


\begin{frame}{A lower bound (confidential interval) for ${\tt gap}(\hat x^N)$}
\begin{itemize}
\item Choose $M>0$, and randomly generate $M$ samples of size  $N$
\pula
\item Solve $M$ problems SAA to obtain $\hat f^N_i$, $i=1,\ldots, M$
\pula
\item Compute the unbiased estimator of $\E[\hat f^N]$:
\[
\bar f^{N,M} :=\frac 1 M \sum_{i=1}^M \hat f^N_i
\]
\pula
\item Compute the variance of $\bar f^{N,M}$:
\[
\hat \sigma^2_{N,M}:=\frac{1}{M(M-1)}\sum_{i=1}^M[\hat f^{N}_i - \bar f^{N,M} ]^2
\]
\pula
\item Compute the lower bound for the $100(1-\alpha)$-confidential interval  of $\E[\hat f^N]$:
% ${\tt gap}(\hat x^N)$:
\azul{\[
L_N':= \bar f^{N,M} - t_{\alpha,\nu}\hat \sigma_{N,M}\,,
\]
}
where $\nu=M-1$ and $t_{\alpha,s}$ is the critical value of the Student's $t$-distribution with $\nu$ degrees of freedom 
\end{itemize}
\end{frame}
%
\begin{frame}{Evaluating a candidate solution}
Given $ \hat x^N \in X $, solution of a SAA problem, we wish to estimate the optimality gap
\azul{
\[
{\tt gap} (\hat x^N) = f (\hat x^N) -f^* \geq 0
\]
}
\begin{block}{Upper bound}
In order to calculate a statistical upper bound  $ U_{N '} (\hat x^N) $ for $ {\tt gap} (\hat x^N) $  we ``only" need to evaluate $ f^{N'}( x^N) $ and calculate its variance
\azul{
\[
U_{N}'(\hat x^N):= f^{N'}(\hat x^N) + z_{\alpha}\hat \sigma_{N'}(\hat x^N)
\]
}
\end{block}
\pause
\begin{block}{Lower bound}
In order to calculate a statistical lower bound $ L_{N, M}$ to $\E[\hat f^N]$ we need to solve $ M $ SAA problems
and calculate its average and variance
\azul{\[
L_N':= \bar f^{N,M} - t_{\alpha,\nu}\hat \sigma_{N,M}
\]
}
\end{block}
\end{frame}


\begin{frame}{Evaluating a candidate solution}
\[
{\tt gap}(\hat x^N) = f(\hat x^N)-f^*\geq 0
\]

\begin{itemize}
\item We have that
$
\hat{{\tt gap}}(\hat x^N):= U_{N'}(\hat x) - L_{N,M}\geq 0
$. Then,
\[
\azul{[0,\,\hat{{\tt gap}}(\hat x^N)]}
\]
is a $(1-2\alpha)$-confidence interval for the optimality gap  ${\tt gap} (\hat x^N)$

\pula
\item If the estimator $\hat{{\tt gap}}(\hat x^N):= U_{N'}(\hat x) - L_{N,M}$ is small enough, so is the optimality gap  ${\tt gap}(\hat x^N)$
\pula
\item Hence, we can say that $\hat x^N$ is a good candidate for solving the true problem
\[
f^*=\min_{x \in X} f(x)\quad \mbox{with}\quad f(x)=\int_{\omega \in \Omega} f(x,\omega)dP(\omega) 
\]
\end{itemize}
\end{frame}



\begin{frame}{The KS-test}{Kolmogorov-Smirnov test}
\[
\hat x^N \in \arg \min_{x \in X} f^N(x), \quad \quad f^N(x)=\frac{\sum_{i=1}^N f(x,\omega^i)}{N}\quad (SAA)
\]

\begin{itemize}
\item In order to infer if the sample size $N$  is satisfactory we may compare
\pula
\begin{itemize}
\item $f^N(\hat x^N)$ with $f^{N'}(\hat x^N)$   (average of the individual costs $f(\hat x^N, \tilde \omega^j)$)
\pula

\item Empirical distribution of the individual costs  $f(\hat x^N, \tilde \omega^j),\;\;j=1,\ldots,N'$ and $f(\hat x^N,\omega^i),\;\;i=1,\ldots,N$ (KS-test)
\end{itemize}
\end{itemize}

\begin{center}
\includegraphics[width=5cm]{../Figs/KSteste.png} {}
\end{center}
 \end{frame} 
 
\begin{frame}{The KS-test}

Another idea widely used in practice is:
\begin {itemize}
\item Given $ N $ and $ M $, generate $ M $  samples of size $ N $ and solve $ M $ problems $ SAA $
\pula

\item Compare the most important variables of the $ M $  SAA solutions $ \hat x^N_i$, $ i = 1, \ldots, M$
\pula

\item Evaluate the SAA solutions using a larger sample $\{ \tilde \omega^1, \ldots, \tilde \omega^{N'} \}$
\pula

\item Compare the empirical cost distributions
\end{itemize}
\pula

\azul{If there is a certain ``adherence" among the results, the size $ N $ can be considered satisfactory. Otherwise, it is suggested to increase $ N $}

\pula
\verde{Importance of simulation}
\begin{itemize}
\item It allows  us to analyze the quality of solution obtained with the stochastic model
\pula

\item it allows us to estimate an appropriated sample size
\end{itemize}
\end{frame} 

\begin{frame}{Computational practice}
Consider the following 2-SLP\[
 \min_{x \in  X}\,f(x)\quad\mbox{com}\quad f(x):=c^\top  x + \E[Q(x, \xi)]\; \mbox{ with}
\]
\[
 Q(x, \verm{\xi}):=\min_{y} q^{\top}  y \quad \mbox{s.a} \quad Tx + Wy =\verm{\xi},\;\; y\geq 0
\]

In this example, $x,y \in \Re^{60}$, $T,W \in \Re^{40\times 60}$ and $X=\{x\geq0: Ax=b\}$ with $b \in \Re^{30}$
\pula

The random vector $\xi=h(\omega)$ follows a multivariate probability distribution
\pula

The problem's data and scenarios are available at the link \url{www.oliveira.mat.br/teaching}
\pula

A line of the file {\tt Sample1.csv}  = a scenario $\xi^i$  (first line = first scenario)

\pula Solve the Equivalent deterministic for $N = 5, 10, 1\,000$ and $10\,000$  
%\pula The SAA approximation of this problem is
%\[
% \min_{x \in  X}\,c^\top  x + \frac {1}{N}\sum_{i=1}^NQ(x, \xi^i)
%\]
\end{frame}


\begin{frame}{A numerical example}
{N = 5 scenarios. Simulation N'=1000, M = 10. Sample 1}

\begin{itemize}
\item Lower bound: 628.744 

\item SAA value:  661.549 

\item Simulated value:  705.676 

\item Upper bound:  711.364 
\end{itemize}

\centering \includegraphics[width=1.1\textwidth]{../Figs/gap5a}  
\end{frame}

\begin{frame}{A numerical example}
{N = 5 scenarios. Simulation N'=1000, M = 10. Sample 2}

\begin{itemize}
\item Lower bound: 620.804  

\item SAA value:  669.281 

\item Simulated value:  728.717  

\item Upper bound:  734.633 
\end{itemize}

\centering \includegraphics[width=1.1\textwidth]{../Figs/gap5b}  
\end{frame}


\begin{frame}{A numerical example}
{N = 5 scenarios. Simulation N'=1000, M = 10. Sample 1}
\centering \includegraphics[width=1.1\textwidth]{../Figs/M5a} 
\end{frame}

\begin{frame}{A numerical example}
{N = 5 scenarios. Simulation N'=1000, M = 10. Sample 2}
\centering \includegraphics[width=1.1\textwidth]{../Figs/M5b} 
\end{frame}



\begin{frame}{A numerical example}
{N = 5 scenarios. Simulation N'=1000, M = 10. Sample 1}
\centering \includegraphics[width=1.1\textwidth]{../Figs/ks5a} 
\end{frame}

\begin{frame}{A numerical example}
{N = 5 scenarios. Simulation N'=1000, M = 10. Sample 2}
\centering \includegraphics[width=1.1\textwidth]{../Figs/ks5b} 
\end{frame}



\begin{frame}{A numerical example}
{N = 5 scenarios. Simulation N'=1000, M = 10. Sample 1}
\centering \includegraphics[width=1.1\textwidth]{../Figs/sol5a} 
\end{frame}

\begin{frame}{A numerical example}
{N = 5 scenarios. Simulation N'=1000, M = 10. Sample 2}
\centering \includegraphics[width=1.1\textwidth]{../Figs/sol5b} 
\end{frame}

%--------------------
\begin{frame}{A numerical example}
{N = 100 scenarios. Simulation N'=1000, M = 10. Sample 1}

\begin{itemize}
\item Lower bound: 693.909  

\item SAA value:  693.871 

\item Simulated value:  693.746

\item Upper bound:  698.871 
\end{itemize}

\centering \includegraphics[width=1.1\textwidth]{../Figs/gap100a}  
\end{frame}

\begin{frame}{A numerical example}
{N = 100 scenarios. Simulation N'=1000, M = 10. Sample 2}

\begin{itemize}
\item Lower bound: 695.389  

\item SAA value:  693.353 

\item Simulated value:  695.700  

\item Upper bound:  701.052 
\end{itemize}

\centering \includegraphics[width=1.1\textwidth]{../Figs/gap100b}  
\end{frame}

\begin{frame}{A numerical example}
This means that the optimal value of
\[
 \min_{x \in  X}\,f(x)\,,\quad\mbox{with}\quad f(x):=c^\top  x + \E[Q(x, \xi)],\; 
\]
is withing the interval
\[
[695.389, \; 701.052]
\]
with 90\% of confidence
\end{frame}

\begin{frame}{A numerical example}
{N = 100 scenarios. Simulation N'=1000, M = 10. Sample 1}
\centering \includegraphics[width=1.1\textwidth]{../Figs/M100a} 
\end{frame}

\begin{frame}{A numerical example}
{N = 100 scenarios. Simulation N'=1000, M = 10. Sample 2}
\centering \includegraphics[width=1.1\textwidth]{../Figs/M100b} 
\end{frame}



\begin{frame}{A numerical example}
{N = 100 scenarios. Simulation N'=1000, M = 10. Sample 1}
\centering \includegraphics[width=1.1\textwidth]{../Figs/ks100a} 
\end{frame}

\begin{frame}{A numerical example}
{N = 100 scenarios. Simulation N'=1000, M = 10. Sample 2}
\centering \includegraphics[width=1.1\textwidth]{../Figs/ks100b} 
\end{frame}

\begin{frame}{A numerical example}
{N = 100 scenarios. Simulation N'=1000, M = 10. Sample 1}
\centering \includegraphics[width=1.1\textwidth]{../Figs/sol100a} 
\end{frame}

\begin{frame}{A numerical example}
{N = 100 scenarios. Simulation N'=1000, M = 10. Sample 2}
\centering \includegraphics[width=1.1\textwidth]{../Figs/sol100b} 
\end{frame}
%

\begin{frame}{Further references}
  \begin{itemize}
    \item Shapiro, A., Dentcheva, D., \& Ruszczynski, A. (2021).\\
      \emph{Lectures on stochastic programming: modeling and theory.}
  \end{itemize}
\end{frame}


\begin{frame}[noframenumbering,plain]{ }
%\begin{center}
%\large
%\ovalbox{Thank you very much!}
%\end{center}
\pula

\begin{block}{Contact}
\begin{itemize}
\item [\Letter] \url{ welington.oliveira@mines-paristech.fr}
\item [\Keyboard] \url{www.oliveira.mat.br}
\end{itemize}
\end{block}
\pula


\begin{center}
\includegraphics[width=8cm]{../Figs/LogoPSL}
\end{center}
\end{frame}

\end{document}

