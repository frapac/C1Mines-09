\begin{frame}{Stochastic programs with recourse}
Let's consider SP of the form
\[
\min_{x \in X}\, \E[f(x,\omega)]
\]
where
\begin{itemize}
\item $f:\Re^n\times \Omega \to \Re$ is convex on $x$ (decision variable)
\item $X \subset \R^n$ is a deterministic set (e.g. a fixed polyhedron)
\item $\omega$ is a random vector and $(\Omega,\mathcal{F},P)$ is its probability space
\item $\E[\cdot]$ is the expected value w.r.t. the probability measure $P$
\end{itemize}
\end{frame}


\begin{frame}{Representation of the uncertainties}
\begin{block}{Continuous probability distribution}
\begin{itemize}
\item Sample space  $\Omega$ contains infinitely many elements
\[
\min_{x \in X}\, \E[f(x,\omega)]
\]

\item  For computational reasons, it is necessary to  consider finitely many scenarios $\omega^i \in \Omega$, with associated probability $p_i>0$
\pula

\item Resulting problem
\[
\min_{x \in X}\, f^N(x)\quad \mbox{with}\quad f^N(x):=\sum_{i=1}^{N}p_if(x,\omega^i)
\]
\end{itemize}
\end{block}
\begin{block}{Sample Average Approximation - SAA}
\[
\min_{x \in X}\, \frac{1}{N}\sum_{i=1}^{N}f(x,\omega^i)
\]
\end{block}
 \end{frame}




% \begin{frame}
% \begin{block}{How to proceed when we do not know $P$?}
% \begin{itemize}
% \item In many applications the probability distribution is not precisely known
% \pula

% \item In these cases, $ P $ is estimated by using the historical of the stochastic vector
% \pula

% \item Scenario generation can be done via Monte Carlo simulation (it is advisable not to use
% the historical as scenarios)
% \end{itemize}
% \pula
% \end{block}
% \begin{block}{Representation of the uncertainties}
% \begin{center}
% \includegraphics[width=5cm]{../Figs/formatos.pdf} {}
% \end{center}
% \end{block}
% \end{frame}















\begin{frame}{Evaluating a candidate solution }

The point $\hat x^N$ solution  of
\[
(SAA)\quad \hat f^N=\min_{x \in X} f^N(x)\quad \mbox{with}\quad f^N(x)=\frac{\sum_{i=1}^N f(x,\omega^i)}{N}
\]
is a candidate solution to the ``true" problem
\[
f^*=\min_{x \in X} f(x)\quad \mbox{with}\quad f(x)=\E[f(x,\omega)]
\]
\pula

As the feasible set of both problems are the same, we get \azul{${f}(\hat x^N)\geq f^*$}

\pula
We thus have the following \verm{unknown} optimality gap
\[
{\tt gap} (\hat x)=f(\hat x^N)- f^*\geq 0
\]

\azul{In what follows we are going to find an estimator $\hat{{\tt gap}}(\hat x^N) $ for ${\tt gap} (\hat x^N)$}
 \end{frame}


\begin{frame}{An upper bound (confidential interval) for ${\tt gap}(\hat x^N)$}
\begin{itemize}
\item Generate randomly, from the probability distribution of $\omega$, a (iid) sample with $N$ scenarios
\pula
\item Obtain  $\hat x^N$ a solution of the resulting SAA problem
\pula
\item Generate randomly, from the probability distribution of $\omega$, another (iid) sample with $N'>>N$ scenarios
\pula
\item Compute $f^{N'}(\hat x^N)= \frac{1}{N'}\sum_{i=1}^{N'} f(\hat x^N,\omega^i)$
\pula
\item Compute the variance of $f^{N'}(\hat x)$
\[
\hat \sigma_{N'}^2(\hat x^N):=\frac{1}{N'(N'-1)}\sum_{i=1}^{N'} [f(\hat x^N,\omega^i)-f^{N'}(\hat x^N)]^2
\]
\item Compute the upper bound for the  $100(1-\alpha)$-confidential interval of $f(\hat x^N)$:
\azul{
\[
U_{N}'(\hat x^N):= f^{N'}(\hat x^N) + z_{\alpha}\hat \sigma_{N'}(\hat x^N)\,,
\]
}
where $z_\alpha=\Phi^{-1}(1-\alpha)$ and $\Phi(z)$ is standard normal distribution. Ex: if $\alpha =5\%$, then $z_\alpha\approx 1.64$.
\end{itemize}
\end{frame}

\begin{frame}{ }
\begin{itemize}
\item Before defining a lower bound for the confidential interval of  ${\tt gap} (\hat x^N)$, notice that, for every given $x \in X$
\[
f(x)=\E[f^N(x)]\geq \E[\min_{y \in X} f^N(y)]=\E[\hat f^N]
\]
(\azul{because  $f^N(x)$ is an unbiased estimator of $f(x)$})
\pula

\item Hence, we get the following useful inequality
\[
f^*\geq \E[\hat f^N]
\]
\pula
\item \azul{We can estimate $\E[\hat f^N]$ by solving several SAA problems}
\end{itemize}
\end{frame}


\begin{frame}{A lower bound (confidential interval) for ${\tt gap}(\hat x^N)$}
\begin{itemize}
\item Choose $M>0$, and randomly generate $M$ samples of size  $N$
\pula
\item Solve $M$ problems SAA to obtain $\hat f^N_i$, $i=1,\ldots, M$
\pula
\item Compute the unbiased estimator of $\E[\hat f^N]$:
\[
\bar f^{N,M} :=\frac 1 M \sum_{i=1}^M \hat f^N_i
\]
\pula
\item Compute the variance of $\bar f^{N,M}$:
\[
\hat \sigma^2_{N,M}:=\frac{1}{M(M-1)}\sum_{i=1}^M[\hat f^{N}_i - \bar f^{N,M} ]^2
\]
\pula
\item Compute the lower bound for the $100(1-\alpha)$-confidential interval  of $\E[\hat f^N]$:
% ${\tt gap}(\hat x^N)$:
\azul{\[
L_N':= \bar f^{N,M} - t_{\alpha,\nu}\hat \sigma_{N,M}\,,
\]
}
where $\nu=M-1$ and $t_{\alpha,s}$ is the critical value of the Student's $t$-distribution with $\nu$ degrees of freedom
\end{itemize}
\end{frame}
%
\begin{frame}{Evaluating a candidate solution}
Given $ \hat x^N \in X $, solution of a SAA problem, we wish to estimate the optimality gap
\azul{
\[
{\tt gap} (\hat x^N) = f (\hat x^N) -f^* \geq 0
\]
}
\begin{block}{Upper bound}
In order to calculate a statistical upper bound  $ U_{N '} (\hat x^N) $ for $ {\tt gap} (\hat x^N) $  we ``only" need to evaluate $ f^{N'}( x^N) $ and calculate its variance
\azul{
\[
U_{N}'(\hat x^N):= f^{N'}(\hat x^N) + z_{\alpha}\hat \sigma_{N'}(\hat x^N)
\]
}
\end{block}
\pause
\begin{block}{Lower bound}
In order to calculate a statistical lower bound $ L_{N, M}$ to $\E[\hat f^N]$ we need to solve $ M $ SAA problems
and calculate its average and variance
\azul{\[
L_N':= \bar f^{N,M} - t_{\alpha,\nu}\hat \sigma_{N,M}
\]
}
\end{block}
\end{frame}


\begin{frame}{Evaluating a candidate solution}
\[
{\tt gap}(\hat x^N) = f(\hat x^N)-f^*\geq 0
\]

\begin{itemize}
\item We have that
$
\hat{{\tt gap}}(\hat x^N):= U_{N'}(\hat x) - L_{N,M}\geq 0
$. Then,
\[
\azul{[0,\,\hat{{\tt gap}}(\hat x^N)]}
\]
is a $(1-2\alpha)$-confidence interval for  ${\tt gap} (\hat x^N)$

\pula
\item If the estimator $\hat{{\tt gap}}(\hat x^N):= U_{N'}(\hat x) - L_{N,M}$ is small enough, so is ${\tt gap}(\hat x^N)$
\pula
\item Hence, we can say that $\hat x^N$ is a good candidate for solving the true problem
\[
f^*=\min_{x \in X} f(x)\quad \mbox{with}\quad f(x)=\int_{\omega \in \Omega} f(x,\omega)dP(\omega)
\]
\end{itemize}
\end{frame}



\begin{frame}{The KS-test}{Kolmogorov-Smirnov test}
\[
\hat x^N \in \arg \min_{x \in X} f^N(x), \quad \quad f^N(x)=\frac{\sum_{i=1}^N f(x,\omega^i)}{N}\quad (SAA)
\]

\begin{itemize}
\item In order to infer if the sample size $N$  is satisfactory we may compare
\pula
\begin{itemize}
\item $f^N(\hat x^N)$ with $f^{N'}(\hat x^N)$   (average of the individual costs $f(\hat x^N, \tilde \omega^j)$)
\pula

\item Empirical distribution of the individual costs  $f(\hat x^N, \tilde \omega^j),\;\;j=1,\ldots,N'$ and $f(\hat x^N,\omega^i),\;\;i=1,\ldots,N$ (KS-test)
\end{itemize}
\end{itemize}

\begin{center}
\includegraphics[width=5cm]{../Figs/KSteste.png} {}
\end{center}
 \end{frame}

\begin{frame}{The KS-test}

Another idea widely used in practice is:
\begin {itemize}
\item Given $ N $ and $ M $, generate $ M $  samples of size $ N $ and solve $ M $ problems $ SAA $
\pula

\item Compare the most important variables of the $ M $  SAA solutions $ \hat x^N_i$, $ i = 1, \ldots, M$
\pula

\item Evaluate the SAA solutions using a larger sample $\{ \tilde \omega^1, \ldots, \tilde \omega^{N'} \}$
\pula

\item Compare the empirical cost distributions
\end{itemize}
\pula

\azul{If there is a certain ``adherence" among the results, the size $ N $ can be considered satisfactory. Otherwise, it is suggested to increase $ N $}

\pula
\verde{Importance of simulation}
\begin{itemize}
\item It allows  us to analyze the quality of solution obtained with the stochastic model
\pula

\item it allows us to estimate an appropriated sample size
\end{itemize}
\end{frame}

\begin{frame}{Computational practice}
Consider the following 2-SLP\[
 \min_{x \in  X}\,f(x)\quad\mbox{com}\quad f(x):=c^\top  x + \E[Q(x, \xi)]\; \mbox{ with}
\]
\[
 Q(x, \verm{\xi}):=\min_{y} q^{\top}  y \quad \mbox{s.a} \quad Tx + Wy =\verm{\xi},\;\; y\geq 0
\]

In this example, $x,y \in \Re^{60}$, $T,W \in \Re^{40\times 60}$ and $X=\{x\geq0: Ax=b\}$ with $b \in \Re^{30}$
\pula

The random vector $\xi=h(\omega)$ follows a multivariate probability distribution
\pula

The problem's data and scenarios are available at the link \url{www.oliveira.mat.br/teaching}
\pula

A line of the file {\tt Sample1.csv}  = a scenario $\xi^i$  (first line = first scenario)

\pula Solve the Equivalent deterministic for $N = 5, 10, 1\,000$ and $10\,000$
%\pula The SAA approximation of this problem is
%\[
% \min_{x \in  X}\,c^\top  x + \frac {1}{N}\sum_{i=1}^NQ(x, \xi^i)
%\]
\end{frame}


\begin{frame}{A numerical example}
{N = 5 scenarios. Simulation N'=1000, M = 10. Sample 1}

\begin{itemize}
\item Lower bound: 628.744

\item SAA value:  661.549

\item Simulated value:  705.676

\item Upper bound:  711.364
\end{itemize}

\centering \includegraphics[width=1.1\textwidth]{../Figs/gap5a}
\end{frame}

\begin{frame}{A numerical example}
{N = 5 scenarios. Simulation N'=1000, M = 10. Sample 2}

\begin{itemize}
\item Lower bound: 620.804

\item SAA value:  669.281

\item Simulated value:  728.717

\item Upper bound:  734.633
\end{itemize}

\centering \includegraphics[width=1.1\textwidth]{../Figs/gap5b}
\end{frame}


\begin{frame}{A numerical example}
{N = 5 scenarios. Simulation N'=1000, M = 10. Sample 1}
\centering \includegraphics[width=1.1\textwidth]{../Figs/M5a}
\end{frame}

\begin{frame}{A numerical example}
{N = 5 scenarios. Simulation N'=1000, M = 10. Sample 2}
\centering \includegraphics[width=1.1\textwidth]{../Figs/M5b}
\end{frame}



\begin{frame}{A numerical example}
{N = 5 scenarios. Simulation N'=1000, M = 10. Sample 1}
\centering \includegraphics[width=1.1\textwidth]{../Figs/ks5a}
\end{frame}

\begin{frame}{A numerical example}
{N = 5 scenarios. Simulation N'=1000, M = 10. Sample 2}
\centering \includegraphics[width=1.1\textwidth]{../Figs/ks5b}
\end{frame}



\begin{frame}{A numerical example}
{N = 5 scenarios. Simulation N'=1000, M = 10. Sample 1}
\centering \includegraphics[width=1.1\textwidth]{../Figs/sol5a}
\end{frame}

\begin{frame}{A numerical example}
{N = 5 scenarios. Simulation N'=1000, M = 10. Sample 2}
\centering \includegraphics[width=1.1\textwidth]{../Figs/sol5b}
\end{frame}

%--------------------
\begin{frame}{A numerical example}
{N = 100 scenarios. Simulation N'=1000, M = 10. Sample 1}

\begin{itemize}
\item Lower bound: 693.909

\item SAA value:  693.871

\item Simulated value:  693.746

\item Upper bound:  698.871
\end{itemize}

\centering \includegraphics[width=1.1\textwidth]{../Figs/gap100a}
\end{frame}

\begin{frame}{A numerical example}
{N = 100 scenarios. Simulation N'=1000, M = 10. Sample 2}

\begin{itemize}
\item Lower bound: 695.389

\item SAA value:  693.353

\item Simulated value:  695.700

\item Upper bound:  701.052
\end{itemize}

\centering \includegraphics[width=1.1\textwidth]{../Figs/gap100b}
\end{frame}

\begin{frame}{A numerical example}
This means that the optimal value of
\[
 \min_{x \in  X}\,f(x)\,,\quad\mbox{with}\quad f(x):=c^\top  x + \E[Q(x, \xi)],\;
\]
is withing the interval
\[
[695.389, \; 701.052]
\]
with 90\% of confidence
\end{frame}

\begin{frame}{A numerical example}
{N = 100 scenarios. Simulation N'=1000, M = 10. Sample 1}
\centering \includegraphics[width=1.1\textwidth]{../Figs/M100a}
\end{frame}

\begin{frame}{A numerical example}
{N = 100 scenarios. Simulation N'=1000, M = 10. Sample 2}
\centering \includegraphics[width=1.1\textwidth]{../Figs/M100b}
\end{frame}



\begin{frame}{A numerical example}
{N = 100 scenarios. Simulation N'=1000, M = 10. Sample 1}
\centering \includegraphics[width=1.1\textwidth]{../Figs/ks100a}
\end{frame}

\begin{frame}{A numerical example}
{N = 100 scenarios. Simulation N'=1000, M = 10. Sample 2}
\centering \includegraphics[width=1.1\textwidth]{../Figs/ks100b}
\end{frame}

\begin{frame}{A numerical example}
{N = 100 scenarios. Simulation N'=1000, M = 10. Sample 1}
\centering \includegraphics[width=1.1\textwidth]{../Figs/sol100a}
\end{frame}

\begin{frame}{A numerical example}
{N = 100 scenarios. Simulation N'=1000, M = 10. Sample 2}
\centering \includegraphics[width=1.1\textwidth]{../Figs/sol100b}
\end{frame}
