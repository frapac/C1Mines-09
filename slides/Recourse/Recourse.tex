

\begin{frame}{Stochastic programming with recourse}
Consider the LP
\[
\left\{\begin{array}{ll}
\displaystyle \min_{(x,y)\geq 0} & c^\top x+
q^\top y\\
 s.t. & Ax =  b \\&Tx+Wy=h
\end{array}\right.
\]
Now suppose that some (or all) the data $q,T,W,h$ depend on some random vector $\omega$:
\[
\verm{q(\omega),\,T(\omega),\,W(\omega),\,h(\omega)}
\]
Decisions are sequential in nature:
\begin{equation*}
  x \leadsto \omega \leadsto y
\end{equation*}
We shall to give a ``meaning" to the random LP:
$q(\omega)^\top y(\omega)$ is now random!
\[
\left\{\begin{array}{ll}
\displaystyle \min_{(x,y(\omega))\geq 0} & c^\top x+
q(\omega)^\top y(\omega)\\
 s.t. & Ax =  b \\&T(\omega) x+W(\omega)y(\omega)=h(\omega)
\end{array}\right.
\]

\pula
\verm{A manner is to minimize the expected cost}
\end{frame}
\begin{frame}{Two-stage stochastic linear programs with recourse - 2SLP}

Minimizing the present cost \verm{\large +} the expected value of the future costs
\pula

\begin{block}{Optimization on average}
\[
\left\{\begin{array}{ll}
\displaystyle \min_{(x,y(\omega))\geq 0} & c^\top x+
\E[q(\omega)^\top y(\omega)]\\
 s.t. & Ax =  b \\&T(\omega) x+W(\omega)y(\omega)=h(\omega) \; a.s.
\end{array}\right.
\]
(a.s. = almost surely)
\end{block}

\begin{itemize}
  \item $x$ represents the \emph{here-and-now} variables \\ (the decisions we have to make in the present)
  \item $y(\omega)$ represents the \emph{wait-and-see} decisions, a.k.a \emph{recourse} \\ (the decisions we have to make in the future, depending on the scenario)
\item $W(\omega)$ is the matrix of recourse, and the matrix of technologies $T(\omega)$ couples the variables $x$ and $y$
\end{itemize}

\end{frame}

\begin{frame}{Finitely many scenarios}
%Consider $N$ scenarios $\xi^i$, $i=1,\ldots, N$, with associated probability $p_i >0$ (e.g. $1/N$)
In two-stage stochastic linear programming problems with finitely many scenarios
\verm{$(q^i,T^i,W^i,h^i)$}, $i=1,\ldots,N$, we wish to solve the high dimensional problem
\begin{block}{Deterministic equivalent}
\[
\left\{
\begin{array}{ll}
\displaystyle \min_{x,y^i} & c^\top x+\sum_{i=1}^N p_i [q^{i\,\top} y^i]\\
\mbox{s.t.} & Ax =  b,\; x\geq 0 \\&T^i x+W^i y^i=h^i,\; y^i\geq 0, \; i=1,\ldots,N\\
\end{array}
\right.
\]

\pula

\pause
\azul{This is a LP with a block-arrowhead structure!}


\[
\left\{
\begin{array}{lllllllllllll}
\min & c^\top x &+ p_1q^{1\,\top} y^1  &+ p_2q^{2\,\top} y^2 &+ \cdots  &+ p_Nq^{N\,\top} y^N \\
\mbox{s.t}& Ax =b\\
&\\
& T^1x           &+ W^1y^1&&&&= h^1 \\
& T^2x           &&+ W^2y^2&&&= h^2 \\
& \vdots         &&&\ddots \\
& T^Nx           &&&&+ W^Ny^N&= h^N \\
&\\
&(x, y)\geq 0
\end{array}
\right.
\]

\end{block}
\end{frame}

\begin{frame}{Some orders of magnitude}
  How fast can we solve a large-scale LP using a state-of-the-art solver?


  \begin{table}
  \begin{tabular}{rrr}
    \hline
    n & m & Solve time (s) \\
    \hline
    85M & 98M & 6,000 \\
    223M & 254M & 66,100 \\
    \hline
  \end{tabular}
  \caption{Time to solve a LP with $n$ variables, $m$ constraints}
  \end{table}

  \vspace{2cm}

  Source:
  \begin{itemize}
    \item Rehfeldt, D., Hobbie, H., Schönheit, D., Koch, T., Möst, D., \& Gleixner, A. (2022).
      \emph{A massively parallel interior-point solver for LPs with generalized arrowhead structure, and applications to energy system models.}
  \end{itemize}


\end{frame}


\begin{frame}{Finitely many scenarios}
In two-stage stochastic linear programming problems with finitely many scenarios
\verm{$(q^i,T^i,W^i,h^i)$}, $i=1,\ldots,N$, we wish to solve the high dimensional problem
\begin{block}{Deterministic equivalent problem}
\[
\left\{
\begin{array}{lllllllllllll}
\min & c^\top x &+ p_1q^{1\,\top} y^1  &+ p_2q^{2\,\top} y^2 &+ \cdots  &+ p_Nq^{N\,\top} y^N \\
\mbox{s.t}& \verm{x \in X, \;y \in Y}  \\
&\\
& T^1x           &+ W^1y^1&&&&= h^1 \\
& T^2x           &&+ W^2y^2&&&= h^2 \\
& \vdots         &&&\ddots \\
& T^Nx           &&&&+ W^Ny^N&= h^N
\end{array}
\right.
\]

\azul{We can have "mixed-integer constraints" in $X$ and $Y$}
\pula

\verm{However, depending on $N$ the deterministic equivalent problem cannot be solved directly...}
\end{block}

\begin{itemize}
\item \# variables: $n_x + N\, n_y$
\item \# constraints: $m_x + N\, n_y$
\end{itemize}


\verm{The deterministic equivalent problem is only useful when $N$ is small enough...}

\end{frame}
%
%\begin{frame}{Computational practice}
%\begin{center}
%\azul{\shadowbox{Power generation planning under uncertainty}}
%\end{center}
%\pula
%Consider the problem available at the link \newline
%{\scriptsize
%\url{https://colab.research.google.com/drive/1mJGxrpmOPnG8cJMlNPxULxj-vUsOg6ql?usp=sharing}
%}
%\pula
%
%\verm{Tasks:}
%\begin{enumerate}
%\item Solve the $N$ scenario subproblems and compute the first-stage solutions $\bar x(\omega^i)$, $i=1,\ldots,N$
%\item Compute the average solution
%\[
%x^{\tt av}= \frac{1}{N}\sum_{i=1}^N \bar x(\omega^i)
%\]
%\item Compute the expected cost of the average solution
%
%\item Compute a solution $x^{\tt SP}$ by solving the equivalent deterministic problem
%
%\item Compare $x^{\tt SP}$, $x^{\tt av}$ and their expected cost
%\end{enumerate}
%
%\end{frame}


\begin{frame}{Two-stage stochastic linear programs with recourse}

\begin{center}
\verm{The deterministic equivalent problem can be too large to be solved directly}
\end{center}

\begin{center}
\shadowbox{We need to decompose the problem}
\end{center}

For convenience of notation, let's
write the 2SLP more compactly as
\[
\left\{\begin{array}{ll}
\displaystyle \min_{(x,y)\geq 0} & c^\top x+
\E[q^\top y]\\
 s.t. & Ax =  b \\&T x+Wy=h \;\;\; a.s.
\end{array}\right.
\]
We define also the random vector
\[
\xi(\omega): = (q(\omega),h(\omega),T(\omega),W(\omega))
\quad \mbox{or simply} \quad
\azul{\xi: = (q,h,T,W)}
\]
and split the problem according to first and second-stage variables

\end{frame}



\begin{frame}{Two-stage decomposition}
  \begin{block}{Value function}
    For each pair \verm{$(x,\xi)$}, define the simple LP
    \[Q(x,\xi):=\left\{ \begin{array}{rll}
        \min & q^\top y &\\
        \mbox{s.t.} & W y   =h-Tx&\\
                    & y  \geq 0  &
    \end{array} \right.\]
  \end{block}
The program 2SLP is equivalent to
\[\left\{ \begin{array}{rll}
\min & c^\top x +\E\left[Q(x,\xi)\right] &\\
\mbox{s.t.} & Ax=b\,, \quad x\geq0\,,&
\end{array} \right.\]

\begin{itemize}
\item For fixed $\tilde\xi$, when is $Q(\cdot,\tilde\xi)$ finite?
\item What does $Q(x,\tilde\xi)=-\infty$ mean?
\only<2>{ \verm{(no solution - depending on $\prob(\tilde \xi)$)}}
\item What does $Q(x,\tilde\xi)=+\infty$ mean?
\only<2>{\verm{(infeasibility!)}}
\end{itemize}



\end{frame}

\begin{frame}{Well-posedness}
  \begin{block}{Fixed recourse}
    The two-stage problem has \emph{fixed} recourse if $W$ does
    not depend on $\omega$.
  \end{block}

  \begin{block}{Complete recourse}
    The two-stage problem has \emph{complete} recourse if the system
    $W y = z$ has a solution for every $z$ (ensure feasibility for all $z := h - Tx$)
  \end{block}

  \begin{block}{Relatively complete recourse}
    The two-stage problem has \emph{relatively complete} recourse
    if for every $x$ in the set $\{ x \; : \; Ax = b \;, \; x \geq 0 \}$
    and for every $\xi \in \Xi$ the feasible set is non-empty:
    \begin{equation*}
     Y(x, \xi) \neq \emptyset
    \end{equation*}
    with
    \begin{equation*}
      Y(x, \xi) = \{ y \; : \; Tx + Wy = h \;, \; y \geq 0 \}
    \end{equation*}
  \end{block}
\end{frame}

\begin{frame}{Properties of the value function}
  The dual of the second-stage problem is
  \begin{equation*}
      \min_{u} \; u^\top (h - Tx) \quad \text{s.t.} \quad W^\top u \leq q
  \end{equation*}

  \begin{block}{Structure}
    \begin{itemize}
      \item The function $Q(\cdot, \xi)$ is convex.
      \item If $\{u \; : \; W^\top u \leq q \}$
        is non-empty and second-stage problem is feasible,\\
        then $Q(\cdot, \xi)$ is \emph{polyhedral}.
    \end{itemize}
  \end{block}

  \begin{block}{Dual reformulation}
    Suppose $Y(x, \xi) \neq \emptyset$. Then
    \begin{equation*}
      Q(x, \xi) = \max_{u} \; u^\top (h - Tx) \quad \text{s.t.} \quad W^\top u \leq q
    \end{equation*}
  \end{block}

  \begin{block}{Subdifferentiability}
    Suppose for $(x, \xi)$ $Q(x, \xi) < +\infty$. Then $Q(\cdot, \xi)$ is subdifferentiable
    at $x$, with
    \begin{equation*}
      \partial Q(x, \xi) = -T^\top \mathcal{D}(x, \xi) \quad
      \text{where} \quad
      \mathcal{D}(x, \xi) := \argmax_{u \in \Lambda(q)} u^\top (h - Tx)
    \end{equation*}
  \end{block}
\end{frame}

\begin{frame}{Two-stage program with finitely many scenarios}
  Suppose we have a finite number of scenarios $\xi^1, \cdots, \xi^N$.
  Let
  \begin{equation*}
    \phi(x) := \mathbb{E}\big[ Q(x, \xi) \big] = \sum_{i=1}^N p_i Q(x, \xi_i)
  \end{equation*}
  The two-stage program is equivalent to
  \[
    \left\{ \begin{array}{rll}
        \min & c^\top x +\sum_{i=1}^N p_i Q(x,\xi_i) &\\
  \mbox{s.t.} & Ax=b\,, \quad x\geq0\,,&
  \end{array} \right.
  \]

  \begin{block}{Proposition}
    Suppose there exists $x_0$ such that $\phi(x_0) < +\infty$.\\
    Then $\phi(\cdot)$ is polyhedral, and for all $x \in \dom(\phi)$,
    \begin{equation*}
      \partial \phi(x) = \sum_{i=1}^N p_k \partial Q(x, \xi_i)
    \end{equation*}
  \end{block}
\end{frame}



%\begin{frame}{Two-stage decomposition}
%
%\begin{itemize}
%\item In practice, we need to generate a sample of scenarios $\{\xi^1, \ldots, \xi^N\}$, with scenario probability $p_i$,  from the distribution of $\xi$ and approximate
%\[
%\E[Q(x,\xi)] \quad \mbox{ by }\quad \sum_{i=1}^N p_i Q(x,\xi^i)
%\]
%\pula
%
%\item Notice that $Q(x,\xi)$ (for a fixed $x \in {\tt dom} \,Q$) is a random variable as well
%\pula
%
%\item  \azul{The strong law of large numbers} ensures that
%
%\[
%\lim_{N \to \infty} \frac{1}{N}\sum_{i=1}^N Q(x,\xi^i) = \E[Q(x,\xi)]
%\]
%\verde{with probability 1 - w.p.1}
%
%\pula
%
%\item This result justifies why we can replace $\E[Q(x,\xi)]$ with $\sum_{i=1}^N p_i Q(x,\xi^i)$: if your random scenario generator is good enough, just pick up a ``large enough" $N$ and set $p_i=1/N$
%
%\end{itemize}
%\end{frame}
